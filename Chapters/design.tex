%\addcontentsline{toc}{chapter}{Development Process}
\chapter{Design} % You should concentrate on the more important aspects of the design. It is essential that an overview is presented before going into detail. As well as describing the design adopted it must also explain what other designs were considered and why they were rejected.The design should describe what you expected to do, and might also explain areas that you had to revise after some investigation.Typically, for an object-oriented design, the discussion will focus on the choice of objects and classes and the allocation of methods to classes. The use made of reusable components should be described and their source referenced. Particularly important decisions concerning data structures usually affect the architecture of a system and so should be described here.How much material you include on detailed design and implementation will depend very much on the nature of the project. It should not be padded out. Think about the significant aspects of your system. For example, describe the design of the user interface if it is a critical aspect of your system, or provide detail about methods and data structures that are not trivial. Do not spend time on long lists of trivial items and repetitive descriptions. If in doubt about what is appropriate, speak to your supervisor. You should also identify any support tools that you used. You should discuss your choice of implementation tools - programming language, compilers, database management system, program development environment, etc.Some example sub-sections may be as follows, but the specific sections are for you to define.

@TODO - leading up to the Design section of the report, we should already have established:

* background & objectives
* requirements
* the switch to an agile approach, encompassing design, development and testing all at once.

% ---

As has already been discussed, I opted for a hybrid approach of a plan-driven methodology to begin with, followed by an agile approach for the implementation. The design stage is where one methodology merges into the other.

I invested a lot of time in clarifying requirements, so that I could establish common classes and methods and processes that can be reused, which would have an effect on my design.

I felt that certain design documentation, such as a database schema, would be worth spending time creating and could be designed up front. Other things, such as class diagrams would not be suitable to be designed upfront, since I'd be refactoring my code throughout the process and it would fall out of line with the documentation.

\section{Overall Architecture}

\section{Some detailed design}

\subsection{Even more detail}

\section{User Interface}

\section{Other relevant sections}