\chapter{Implementation}

The implementation should look at any issues you encountered as you tried to implement your design. During the work, you might have found that elements of your design were unnecessary or overly complex; perhaps third party libraries were available that simplified some of the functions that you intended to implement. If things were easier in some areas, then how did you adapt your project to take account of your findings? It is more likely that things were more complex than you first thought. In particular, were there any problems or difficulties that you found during implementation that you had to address? Did such problems simply delay you or were they more significant? You can conclude this section by reviewing the end of the implementation stage against the planned requirements.

\section{Comments}

@TODO mention this somewhere:

The codebase is liberally commented throughout, using API-style markup. This was a difficult decision, discussed and justified in appendix~\ref{appendix:comments}.

\section{SmartResolution directory structure}

As the implementation followed an agile methodology, the design evolved over time and thus, the directory structure could not be determined up-front. Most (but not all) of the key directories and folders are outlined below:

\dirtree{%
.1 data/.
.1 features/.
.1 test/.
.1 vendor/.
.1 webapp/.
.2 core/.
.3 api/.
.3 controller/.
.3 db/.
.3 model/.
.3 view/.
.2 modules/.
.3 other/.
.3 config.json.
.2 uploads/.
.2 index.php.
.2 routes.php.
.1 .travis.yml.
.1 composer.json.
.1 Gemfile.
}

\lstinline{data} contains fixture data for tests. This is also where the test and production SQLite3 databases reside.

\lstinline{features} contains the Cucumber features and Ruby step definitions.

\lstinline{test} contains all PHP unit tests.

\lstinline{vendor} is an automatically generated directory, created by Composer, containing all of SmartResolution's dependencies.

\lstinline{webapp/core} contains the core ODR platform, which uses an MVCR compound design pattern (\lstinline{webapp/routes.php} defines the routing component). The \lstinline{model}, \lstinline{view} and \lstinline{controller} directories are self-explanatory.

Also inside the core is the \lstinline{db} directory, which contains middleware classes connecting the model classes to the database, since models should encapsulate the concept of whatever it is they are representing, rather than being responsible for the relational database to object mapping.

Finally, this folder also contains an \lstinline{api} directory, which defines all of the global functions available to modules. Having these in their own directory made generating module-specific API documentation easy.

Going back up a level, we have \lstinline{webapp/modules}, which contains any installed SmartResolution modules. This is where the maritime collision module resides once it has been installed. A \lstinline{config.json} file (edited in a user-friendly way through the admin dashboard) denotes which modules are installed and whether or not they are active.

Finally, at the top level we have a few interesting files:

\lstinline{.travis.yml} - an instructions file for Travis Continuous Integration, describing how to set up the project and run its tests.

\lstinline{composer.json} - describes SmartResolution's dependencies. Developers can install all dependencies simply by running \lstinline{\$ composer install}.

\lstinline{Gemfile} - describes SmartResolution's Ruby dependencies. Required for the Cucumber and Ruby integration tests.