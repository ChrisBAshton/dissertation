\chapter{Implementation}

\section{Third-party service configuration}

Early on in the project implementation, a few days were spent setting up and utilising third-party services to cut down the costs of maintenance further on in the project's lifecycle. Even in the very early stages when SmartResolution had just one or two unit tests, time was spent configuring a Travis file so that the tests would automatically be run on Travis' CI platform after every pushed commit.

Dependencies such as F3 and PHPUnit were important for the implementation of the project, but they should not exist in the project repository. They also should not add unnecessary complexity to the installation instructions, as developers shouldn't have to manually fulfil each project dependency. For these reasons, I utilised Composer: a dependency manager for PHP which makes it easy for me to specify the project dependencies whilst also making it easy for other developers to install those dependencies. For the same reasons, I utilised RubyGems to specify the Ruby dependencies for the Ruby and Cucumber tests.

Dependency tracking is important, as newer versions of dependencies fix bugs and vulnerabilities discovered in older versions. By not updating SmartResolution's dependencies regularly I would risk crackers being able to exploit unpatched vulnerabilities. This is especially relevant since SmartResolution is open-source and anybody can see which version of which third-party library SmartResolution is using. For these reasons, I signed up to the Gemnasium dependency-monitoring service, which warns me whenever my project's dependencies are out of date.

An important thing to monitor with each commit is code quality, something that CodeClimate is trying to automate. Whenever I push a commit to SmartResolution, CodeClimate queues a scan of the codebase and informs me on a level of 1-4 whether the quality of my codebase has improved or deteriorated, using metrics such as variable name length, code duplication, the number of possible paths through a block of code, and so on. Utilising this third-party service helped fight the temptation to hack a bit of functionality in, instead encouraging good engineering practice and providing suggestions as to how the codebase could be made more maintainable.

\section{SmartResolution directory structure}

As the implementation followed an agile methodology, the design evolved over time and thus, the directory structure could not be determined up front. Most (but not all) of the key directories and folders are outlined below:

\begin{minipage}{\textwidth}
\dirtree{%
.1 data/.
.1 features/.
.1 test/.
.1 vendor/.
.1 webapp/.
.2 core/.
.3 api/.
.3 controller/.
.3 db/.
.3 model/.
.3 view/.
.2 modules/.
.3 other/.
.3 config.json.
.2 uploads/.
.2 .htaccess.
.2 index.php.
.2 routes.php.
}
\end{minipage}

\lstinline{data} contains fixture data for tests. This is also where the test and production SQLite3 databases reside.

\lstinline{features} contains the Cucumber features and Ruby step definitions.

\lstinline{test} contains all PHP unit tests.

\lstinline{vendor} is an automatically generated directory, created by Composer, containing all of SmartResolution's PHP dependencies.

\lstinline{webapp/core} contains the core ODR platform, which uses an MVCR composite design pattern. The \lstinline{model}, \lstinline{view} and \lstinline{controller} directories are self-explanatory and \lstinline{webapp/routes.php} defines the routing component.

Also inside the core is the \lstinline{db} directory, which contains middleware classes connecting the model classes to the database, since models should encapsulate the concept of whatever it is they are representing but not be responsible for the relational database to object mapping. This is discussed in detail later on in this chapter.

This folder also contains an \lstinline{api} directory, which defines all of the global functions available to modules. Having these in their own directory made generating module-specific API documentation easy.

Going back up a level, we have \lstinline{webapp/modules}, which contains any installed SmartResolution modules, such as the maritime collision module. A \lstinline{config.json} file (generated in a user-friendly way through the admin dashboard) denotes which modules are installed and whether or not they are active.

Finally, the \lstinline{.htaccess} server configuration file ensures that all HTTP requests are routed to \lstinline{index.php}, which is the driver for the application.

\section{Routing}

\begin{figure}[h!]
  \centering
    \ifimages
    \includegraphics[width=\textwidth]{routing}
    \fi
  \caption{How SmartResolution processes HTTP requests}
  \label{uml:routing}
\end{figure}

Figure~\ref{uml:routing} shows how SmartResolution routes HTTP requests and renders data-driven pages. As described in the image, the HTTP request is processed by \lstinline{routes.php} and forwarded to the appropriate controller, which then instantiates the models and renders the view.

\section{Class diagrams}

\begin{figure}[h!]
  \centering
    \ifimages
    \includegraphics[width=\textwidth]{class}
    \fi
  \caption{Class diagram showing the model classes in SmartResolution}
  \label{uml:class}
\end{figure}

The system was developed in an agile way, hence these class diagrams are not in the design section but the implementation section. The class diagram in figure~\ref{uml:class} shows mainly the classes in the \lstinline{webapp/core/model} directory, as these models encapsulate the business logic of the application. The following subsections describe the class diagram in more detail.

\subsection{Accounts}

The \lstinline{AccountInterface} defines the core methods which all account types need to implement, whether the account is a law firm, a mediator or even an administrator. These methods include \lstinline{getEmail()}, \lstinline{getLoginId()}, \lstinline{__toString()} and so on.

Many of the methods in the interface will have the same implementation regardless of the account type. In these cases, the abstract class \lstinline{Account} defines the common implementations, but it does not implement the \lstinline{AccountInterface} interface since it cannot implement all of the methods for every account type.

The \lstinline{Organisation} and \lstinline{Individual} account types implement the \lstinline{AccountInterface} interface and extend the \lstinline{Account} abstract class to further define any missing function definitions specific to their own account type. Finally, the specialised subclasses inherit from \lstinline{Organisation} or \lstinline{Individual} and override or call parent function definitions where necessary.

\subsection{Disputes}

Every fully instantiated dispute has two dispute parties, each of which is comprised of a law firm, an agent and a dispute summary. A dispute can be associated with infinite items of evidence (uploaded by the agents) and infinite messages sent back and forth between the agents. Every dispute should also have a lifespan: this is more complicated than it might first appear so is discussed in detail in the next subsection.

Disputes have a \emph{state}, which roughly maps to the red box states indicated in figures~\ref{uml:activity:dispute} and~\ref{uml:activity:mediation}. This is discussed in detail in subsection~\ref{subsection:disputeState}.

A dispute's type denotes whether or not a module is pulled in to expand the options available to it. By default, all disputes are of type `Other', which adds nothing to the core functionality of the system. However, with the maritime collision module activated, agents have the option to change the dispute type to `Maritime Collision', thereby unlocking the AI that the module offers.

Finally, disputes have a \emph{mediation state}: that is, a dispute may or may not be in mediation, or it may be somewhere in between. For example, the agents may have decided upon a mediation centre but not yet chosen a mediator. This is all encapsulated in the \lstinline{MediationState} class.

\subsection{Lifespans}

Every dispute should have a lifespan. This becomes somewhat complicated, since a lifespan can be proposed by either party, must be agreed by both parties, must only be applied between the start and end points of a lifespan, and can be renegotiated at any time.

With this in mind, every dispute can be said to have two lifespans: the `current' and `latest' lifespans. Figure~\ref{uml:lifespan} highlights the difference between the two.

\begin{figure}[h!]
  \centering
    \ifimages
    \includegraphics[width=\textwidth]{lifespan}
    \fi
  \caption{Visualisation showing the difference between the `current' and `latest' lifespans}
  \label{uml:lifespan}
\end{figure}

The `current' lifespan is the most recent accepted Lifespan that has been attributed to the given dispute. If no Lifespan has been accepted, this retrieves the most recent \emph{offered} Lifespan. If no Lifespan has even been offered, this returns a mock Lifespan object so that all of the lifespan-related method calls still work, saving us from having to complicate our dispute object.

For most purposes, the `current' lifespan is what is required. This is the lifespan that has generally been agreed by both parties and is used for checking if a dispute is still ongoing before allowing an agent to send a message, for example.

The `latest' lifespan ignores whether or not a lifespan has been accepted and retrieves the very latest lifespan proposal. This is required in special cases, such as in the rendering of the lifespan on the dispute dashboard, to highlight the fact that a new lifespan has been proposed and needs to be accepted or declined.

\subsection{Dispute State}\label{subsection:disputeState}

Separately from the `mediation state' is the higher level dispute state. A dispute is in a different state depending on whether or not all agents have been assigned, a lifespan has been negotiated, and so on. The state of the dispute dictates what actions are available to the parties involved.

Early versions of the project began to get quite complicated because classes throughout the system would manually determine a dispute's state when deciding whether or not they could perform some action. This led to complicated queries like the one below:

\begin{lstlisting}[language=php]
if ($dispute->state() === 'Open' || $dispute->state() === 'InMediation' || $dispute->state() === 'NegotiatingLifespan') {
    doSomething();
}
\end{lstlisting}

\begin{figure}[h!]
  \centering
    \ifimages
    \includegraphics[width=\textwidth]{states_sequence}
    \fi
  \caption{Sequence diagram showing the state pattern in SmartResolution}
  \label{uml:states}
\end{figure}

Figure~\ref{uml:states} shows how the state pattern was adopted to overcome problems like this. Classes throughout SmartResolution could now query the dispute's state directly, as the responsibility of whether or not an action could be performed was encapsulated inside the state itself. The above example could thus be rewritten as follows:

\begin{lstlisting}[language=php]
if ($dispute->state()->canDoSomething()) {
    doSomething();
}
\end{lstlisting}

The list of possible states are as follows:

\begin{itemize}
    \item \textbf{DisputeCreated} - this is the very first state of the dispute and represents a dispute that has just been created. At this stage, the first dispute party is complete (it will have a law firm and an agent associated with it), but it has not been opened against another law firm.
    \item \textbf{DisputeAssignedToLawFirmB} - this represents the state of the dispute when it has just been assigned to the other law firm. At this stage, one dispute party is complete, whilst the other only has the law firm. We are still waiting for the law firm to assign an agent.
    \item \textbf{DisputeOpened} - all law firms and agents have been assigned. Now a lifespan must be negotiated.
    \item \textbf{LifespanNegotiated} - the agents have managed to negotiate a lifespan and there is nothing more to do to initiate the dispute. When the start date is surpassed, the dispute is underway and the agents are free to perform all dispute-related actions. When the end date passes, the dispute is automatically closed.
    \item \textbf{DisputeInMediation} - the agents have decided to put the dispute into mediation and have negotiated a mediation centre and a mediator. It is important to note that not all disputes will necessarily reach this stage.
    \item \textbf{DisputeInRoundTableMediation} - the dispute is in mediation, but all parties are free to communicate openly. By default, a dispute in mediation disables direct communication between the two agents. The mediator can enable round-table communication to put the dispute into this state.
    \item \textbf{DisputeClosed} - the dispute is now closed, either because an agent closed it or because the lifespan of the dispute came to an end. It may have been closed successfully (the dispute was resolved) or unsuccessfully (the dispute had to be resolved by other means, e.g. court).
\end{itemize}

As can be seen in the class diagram in figure~\ref{uml:class}, most of these states inherit from the DisputeStateDefaults class, which defines the default permissions regarding dispute actions. Inheriting from this class means we don't have to specify long lists of true/false values where only one or two items may have changed between states. The exception to this rule is the \lstinline{InRoundTableMediation} state, which is only very slightly different to the \lstinline{InMediation} state and thus extends that class rather than the default class.

\section{Refactoring to pure models}
\label{section:refactoringToPureModels}

\subsection{Problem with original design}

The early design had all models populated with an ID that corresponded to a row of data in the database. For example, the \lstinline{Message} constructor expected to be passed an ID corresponding to the database field \lstinline{messages.message_id}. Inside the constructor, a function call was made to a data wrapper class which contained the business logic for retrieving the class-specific data (such as message content, author ID, etc) from the database. The returned array was then used to populate the model. This behaviour is suggested in the routing visualisation in figure~\ref{uml:routing}.

The decision was made to implement the models like this to make life easier for the controllers. Let us examine this with some pseudo-code:

\begin{minipage}{\textwidth}
\begin{lstlisting}[language=php]
// inside a controller
$disputeID = getDisputeIDFromUrl();
$dispute = new Dispute($disputeID);
\end{lstlisting}
\end{minipage}

Inside the Dispute model, we had something like this:

\begin{minipage}{\textwidth}
\begin{lstlisting}[language=php]
function __construct($disputeID) {
    $disputeDetails = DisputeDatabaseConnector::getDisputeDetails($disputeID);
    $this->title = $disputeDetails['title'];
    // and so on
}
\end{lstlisting}
\end{minipage}

This implementation was advantageous insofar as I could write simple code in the controllers that only had to worry about getting an object's ID. This meant that I could chain method calls together, passing returned IDs into subsequent constructors to quickly and easily get the objects I needed.

The disadvantage of this implementation was that the models were made impure. Though the models were somewhat decoupled from the database - all SQL queries were encapsulated in the intermediary database querying objects - they still had to know about the existence of the database, and they could not exist independently of a persistence layer.

Database querying was not restricted to the constructors; any side-effect functions\footnote{These are functions which perform some side-effect when called. The internal state of the object changes as a direct result of calling the function.} were also tightly coupled to the database. For example:

\begin{minipage}{\textwidth}
\begin{lstlisting}[language=php]
// inside the Dispute model
function closeSuccessfully) {
    SomeDatabaseConnector::updateField('disputes', 'status', 'resolved', $this->disputeID);
    $this->closed = true;
}
\end{lstlisting}
\end{minipage}

Again, this muddled the responsibilities of the model and made it difficult to test and to maintain. The model was tightly coupled to the underlying table representation, and functions had to know how to update individual fields related to the object.

\subsection{Refactored solution}\label{subsection:refactoredSolution}

The decision was made to change models to take an array of data rather than a database row ID, fully decoupling them from the database. The array of data may or may not have come from a database: the models have no concept of persistence.

Our earlier pseudocode examples can now be rewritten as:

\begin{minipage}{\textwidth}
\begin{lstlisting}[language=php]
// inside the Dispute model
function __construct($disputeDetails) {
    $this->title = $disputeDetails['title'];
    // and so on
}

// inside the controller
$disputeID = getDisputeIDFromUrl();
$disputeDetails = DisputeDatabaseConnector::getDisputeDetails($disputeID);
$dispute = new Dispute($disputeDetails);
\end{lstlisting}
\end{minipage}

\begin{figure}[h!]
  \centering
    \ifimages
    \includegraphics[width=\textwidth]{model_update}
    \fi
  \caption{Demonstration of how changes to the model are made persistent}
  \label{uml:modelUpdate}
\end{figure}

After extracting the database array retrieval out of the model, I extracted the database \emph{updating} out of the model. When moving this behaviour into the controllers, I decided to encapsulate all updates inside one update method that corresponds to the model:

\begin{minipage}{\textwidth}
\begin{lstlisting}[language=php]
// inside the Dispute controller
$dispute->closeSuccessfully();
DBUpdate::instance()->dispute($dispute);
\end{lstlisting}
\end{minipage}

Instead of updating an individual field value, I now pass the entire Dispute object to an update method which knows how to map the dispute object to its representation in the database. It calls the necessary methods on the dispute object (such as \lstinline{$dispute->getStatus()}) to collect all of the information it requires to update the record. This behaviour is demonstrated in figure~\ref{uml:modelUpdate}.

This large refactoring task perhaps hints at a broader problem: object-relational mapping. It could be argued that at various points in my codebase I have not approached the problem in the object-oriented way I should have, because I've had to manually map the object to be in terms of its underlying relational database representation. Perhaps, in hindsight, an object-relational database would have been more appropriate than an RDBMS, to delegate the understanding of the object-relational mapping to a third-party.

\section{Controllers}

Controllers can be split into two main types: handlers and database middleware. The handler in figure~\ref{uml:routing} is deduced by the routing file and is likely to be a controller in \lstinline{core/controller}. Handlers simply implement the behaviour seen in the use case diagrams in chapter~\ref{chapter:requirements} and, as such, a fully comprehensive class diagram was not deemed necessary. However, a demonstration of how these controllers work can be seen in figure~\ref{uml:controllers}.

\begin{figure}[h!]
  \centering
    \ifimages
    \includegraphics[width=\textwidth]{controllers}
    \fi
  \caption{Visualisation of HTTP requests being routed to controller method}
  \label{uml:controllers}
\end{figure}

Database middleware classes live in \lstinline{core/db} and handle the mapping between the models and the database. Each class fulfils a distinct database-querying need:

\begin{itemize}
\item \textbf{Database.php} - links with F3 and provides the database connection.
\item \textbf{DBCreate.php} - responsible for creating the necessary rows in the database for a given object type.
\item \textbf{DBGet.php} - retrieves array values from the database, given a row ID.
\item \textbf{DBUpdate.php} - given an object, it finds the corresponding database entry and updates the values. Used for making object changes persistent.
\item \textbf{DBQuery.php} - defines miscellaneous queries that would not be efficient or cohesive if encapsulated elsewhere.
\end{itemize}

\section{Module API implementation}

\begin{figure}[h!]
  \centering
    \ifimages
    \includegraphics[width=\textwidth]{module_pubsub}
    \fi
  \caption{Visualisation showing how modules are registered to the system}
  \label{uml:moduleRegistration}
\end{figure}

The design for the module API was discussed in the design section. This section discusses how the module API was actually implemented, as demonstrated in figure~\ref{uml:moduleRegistration}.

The configuration JSON file is edited through the admin dashboard using a user-friendly GUI: administrators can activate or deactive modules at the press of a button. This file maintains a list of all known modules in the installation and whether or not they are active.

On every page load, \lstinline{config.php} is called. This iterates through the modules described in \lstinline{config.json} and loads the module's \lstinline{index.php} file, which contains the module declaration call. All modules are pulled in regardless of their activity state, since the system still needs to know about them for the purposes of the admin dashboard.

Note that the module declaration function call is different to the module definition function. The former is always executed, as it declares the module to the system. It \emph{contains} the module definition as an anonymous function which is only executed if the module is `active'.

The module declaration function is defined in \lstinline{api.php} and in turn calls the \lstinline{registerModule} function in the \lstinline{ModuleController}. That function instantiates a new \lstinline{Module} object representing the module and then queries its activity state. If the module is active, its module definition function is called and the module is now hooked into all of the necessary events and global functions.

\subsection{Module definition function}

The module definition function contains calls to the module API. Some of these are top-level, such as defining custom routes: these are executed immediately.

The most interesting API call is the publish-subscribe function \lstinline{on}, first discussed in subsection~\ref{subsection:moduleSupport}. This function calls the \lstinline{ModuleController->subscribe} function, which adds the callback function to an internal array of subscriptions. The position at which the subscription is added to the array is denoted by the priority of the subscription.

If and when an event is emitted by a class in the SmartResolution core platform, any subscriptions to that event are retrieved. They are then iterated through and the following check is performed (demonstrated here in pseudocode):

\begin{lstlisting}[language=python]
	if the event is dispute-agnostic:
	    call the subscribed function

	else if the event DOES rely on a dispute:
	    if the subscribed function module name matches the type of the current dispute:
	        call the subscribed function
\end{lstlisting}

This means that SmartResolution supports both dispute-independent hooks (i.e. ``call this function when this event is emitted") and dispute-dependent hooks (i.e. ``call this function when this event is emitted, but only if the dispute emitting this event is the same type as the name of this module"). An example of a dispute-dependent hook is described in the next subsection.

\subsection{Demonstration of module implementation}

\begin{figure}[h!]
  \centering
    \ifimages
    \includegraphics[width=\textwidth]{screenshot--dispute}
    \fi
  \caption{Screenshot of a dispute populated with fixture data}
  \label{screenshot:dispute}
\end{figure}

\begin{figure}[h!]
  \centering
    \ifimages
    \includegraphics[width=\textwidth]{screenshot--dispute_maritime}
    \fi
  \caption{Screenshot of the same dispute changed to the `maritime collision' dispute type}
  \label{screenshot:disputeMaritime}
\end{figure}

Figure~\ref{screenshot:dispute} is a screenshot of an example dispute. Contrast this with figure~\ref{screenshot:disputeMaritime}, which shows the same dispute but with the dispute type changed from `Other' to `Maritime Collision'. By doing so, when the dispute-dependent \lstinline{dispute_dashboard} event was fired, the maritime collision subscribed function was executed. Inside that function, the \lstinline{dashboard_add_item} function was called, specifying the maritime collision icon, title and link.

\begin{figure}[h!]
  \centering
    \ifimages
    \includegraphics[width=\textwidth]{screenshot--module_maritime}
    \fi
  \caption{Screenshot of one of the questions screens of the maritime collision module}
  \label{screenshot:moduleMaritime}
\end{figure}

The dashboard item links through to the maritime collision landing page which displays domain-specific questions, after both agents have accepted a disclaimer message. One of the questions pages is depicted in the screenshot in figure~\ref{screenshot:moduleMaritime}.

\begin{figure}[h!]
  \centering
    \ifimages
    \includegraphics[width=\textwidth]{screenshot--maritime_module_results}
    \fi
  \caption{Screenshot of a possible results screen of the maritime collision module}
  \label{screenshot:moduleMaritimeResults}
\end{figure}

When all of the questions have been answered, the module's AI interprets the answers to the questions according to maritime law and presents each agent with a personalised result, shown in figure~\ref{screenshot:moduleMaritimeResults}. Underneath the overall result is a summary of all of the questions asked and the answers provided by both agents.

\section{SmartResolution Marketplace}

The SmartResolution website was developed locally using many of the same technologies as the core SmartResolution software, including F3, Bootstrap, Composer and so on. For the SmartResolution Marketplace functionality to be implemented, it was clear that the website would need to be deployed somewhere remotely.

\subsection{Choosing a server}

A locally configured, physical server would not be practical for hosting the SmartResolution marketplace, as it would not be able to cope with a spike in network traffic if there were large numbers of people downloading the software or browsing the documentation. Moreover, it is an unnecessary maintenance and up front expense when so many alternatives exist.

Cloud computing as a web hosting service is becoming increasingly popular, Amazon Web Services (AWS) chief amongst these in terms of exponential growth.~\cite{article:AWS} AWS is being utilised by companies including the BBC, its popularity down to its speed and scalability, as well as the low cost to market.~\cite{article:bbcAws}

AWS has server farms in 11 geographical regions including the US, Ireland, Tokyo and Australia, making it possible to make optimised regional sites or even content delivery networks (CDNs).~\cite{aws:global} It can be configured to automatically deploy additional EC2 (Amazon Elastic Compute Cloud) instances when there is a surge in website traffic, to cope with fluctuations in demand.

Other alternatives exist, of course, such as Unlimited Web Hosting, which is a cloud-based web hosting service I like to use for my personal projects. However, SmartResolution requires shell access for the installation process in order to download dependencies through Composer, set up the SQLite database and so on.

Unlike Unlimited Web Hosting, AWS provides shell access, and though other services also provide this facility (DigitalOcean, for example), it made sense to invest time in configuring an infrastructure that would be able to cope with increases in traffic should SmartResolution ever require it.

\subsection{Configuring the server}

\begin{figure}[h!]
  \centering
    \ifimages
    \includegraphics[width=\textwidth]{server_config}
    \fi
  \caption{The server configuration for smartresolution.org}
  \label{uml:serverConfig}
\end{figure}

At the point where an EC2 instance is started, a dynamic IP address is generated, making the instance available at that given IP. By default, that IP address is not persistent and every 24 hours or so the IP addresses are reallocated and the instance must be reached through a different IP.

AWS offers an `Elastic IP' facility, which allows you to generate a permanent IP address and allocate it to an EC2 instance. This costs a little extra but is a necessary step to ensure the website is always locatable.

Finally, Amazon's Route 53 service is a scalable DNS and allows you to link a domain/subdomain name to an IP address so that browsers querying that domain name are redirected to the content at the IP address endpoint.

smartresolution.org was purchased through the domain registrar gandi.net and the nameservers for Amazon's Route 53 service were specified as the DNS. On the server itself, the EC2 instance is automatically deployed as a LAMP server running Amazon's own flavour of Linux, Apache, MySQL and PHP.~\cite{aws:linux}

The decision was made to separate the two concepts of the SmartResolution software and the SmartResolution website, so a live demo of the software should be available on a subdomain rather than the main site index. To accomplish this, a VirtualHost was specified in the Apache configuration to redirect any requests for demo.smartresolution.org to a specific demo folder containing the SmartResolution software, which could be easily updated independent of SmartResolution website updates. All of this is demonstrated in figure~\ref{uml:serverConfig}.

\subsection{Continuous deployment}

Though continuous integration is important for ensuring that the codebase remains fully functional, continuous deployment is important for ensuring that the version of SmartResolution available for demo and for download on the SmartResolution website is fully up to date. It needed to be simple to keep both elements at the latest version, and everything ought to be as automated as possible.

I wrote a bash script specific to the SmartResolution vendor site which clears the html and html-demo folders outlined in figure~\ref{uml:serverConfig}, then re-downloads and installs the website and the software from the website repository and the software repository respectively. The script calls PHPDoc to generate the SmartResolution core API documentation and module documentation, and the script also strips out unnecessary files such as tests and Travis configuration before zipping up the file as a production-ready download.

In an ideal world, this script would be triggered via a Git web hook so that the website, demo and download file are all updated whenever an update is pushed to either the website or software repository. However, triggering this bash script through PHP is a security issue and is made very difficult to accomplish through AWS' default Apache and PHP configuration. After some failed attempts and with time pressing on, it was decided that manually signing into the EC2 instance and running the update script was not too much of an inconvenience.

\section{Documentation}

\subsection{Comments}

I believe in the agile principles of TDD and user stories as a replacement for documentation, and further believe that any documentation that is not intrinsically linked to the code itself will always eventually become misaligned with it.

For example, SmartResolution uses Cucumber features, which is a form of `executable' documentation. The same can be said of unit tests, though these are less descriptive. Code should be self-documenting, and anything in the code that requires an explanatory comment is a sign that the code itself should be refactored.

All this in mind, I made the difficult decision to write Javadoc-style comments throughout the SmartResolution codebase: this decision is discussed in detail in appendix~\ref{appendix:comments}. Though I'm concerned that even these kind of comments eventually fall out of place with the code that they refer to, they do at least have the advantage of being able to be parsed to generate HTML documentation, which can be deployed to the project website. This ensures that the process of keeping API documentation up-to-date on the website is made as simple as possible, even if the applicability of the comments themselves cannot be fully verified without looking at the code.

SmartResolution API documentation, aimed at contributors to the SmartResolution platform, is available at:

\url{http://smartresolution.org/docs/index.html}

Developers who wish to create SmartResolution \emph{modules} can find module-specific API documentation at the following address:

\url{http://smartresolution.org/module-docs/index.html}

\subsection{Installation \& how-to guides}

The same philosophy cannot be applied to installation instructions or how-to guides, as there is currently no technological implementation that allows these to be generated from the code itself. To maximise the usefulness of SmartResolution and the likelihood of its continued existence, I expended quite some effort in creating high-quality, relevant additional documentation.

Instructions for installing SmartResolution on your own server can be found at the following address:

\url{http://smartresolution.org/installation}

A how-to guide for developing SmartResolution modules can be found at the following address:

\url{http://smartresolution.org/module-how-to}