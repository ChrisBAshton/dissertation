\chapter{Background \& Objectives}

%This section should discuss your preparation for the project, including background reading, your analysis of the problem and the process or method you have followed to help structure your work.  It is likely that you will reuse part of your outline project specification, but at this point in the project you should have more to talk about. 

\section{Introduction} % What was your background preparation for the project? What similar systems did you assess? What was your motivation and interest in this project?

\emph{Online Dispute Resolution for Maritime Collisions} was an idea put forward by Dr Constantina Sampani, a Lecturer in Law at Aberystwyth University, and presented as a Major Project idea through Dr Alexandros Giagkos, a Research Associate in the Computer Science department at the same university. [REF] % https://www.aber.ac.uk/en/law-criminology/staffdirectory/cos21/
% http://www.aber.ac.uk/en/cs/staff-list/staff_profiles/?login=alg25

The idea was to create an online dispute resolution (ODR) platform that is able to suggest a resolution using artificial intelligence (AI). Maritime law was considered to be a good example to demonstrate this, as it was thought to be quite simple, concise and relatively straightforward to translate into code. It was hoped that, with the foundation work laid out in the maritime law business logic, interpretation of more complex laws could be automated in future.

With this in mind, as well delivering the maritime collision AI, it was important to create an abstract ODR platform that could take any arbitrary module of business logic other than just maritime law, so that it could be utilised in increasingly sophisticated ways in future iterations of the project.

\section{Background \& Analysis} % Taking into account the problem and what you learned from the background work, what was your analysis of the problem? How did your analysis help to decompose the problem into the main tasks that you would undertake? Were there alternative approaches? Why did you choose one approach compared to the alternatives? There should be a clear statement of the objectives of the work, which you will evaluate at the end of the work. In most cases, the agreed objectives or requirements will be the result of a compromise between what would ideally have been produced and what was felt to be possible in the time available. A discussion of the process of arriving at the final list is usually appropriate.

\subsection{An introduction to ODR}

Online Dispute Resolution (ODR) is a specialised type of Alternative Dispute Resolution (ADR), which refers to ``any means of settling disputes outside of the courtroom" [reference]. ADR can include negotiation, mediation, arbitration and so on. % reference = https://www.law.cornell.edu/wex/alternative_dispute_resolution

There are various motivations for ODR. Since neither party is required to travel to a physical courtroom, disputes can be settled more quickly, conveniently and at lower cost. Modria, ``The World’s Leading Online Dispute Resolution Experts" [reference], cite three general reasons to use ODR: % reference = http://modria.com/products/

\begin{itemize}
    \item Reduced legal risk – resolving disputes quickly decreases the chance of lawsuits because your customers feel like they’re being heard.
    \item Lower operating costs – managing disputes online lowers travel expenses.
    \item Increase customer loyalty – fast, fair resolutions = happier customers.
\end{itemize}

ODR platforms already exist; they allow lawyers to open disputes on behalf of their clients, upload documents as evidence and communicate via online messaging, hopefully reaching an amicable resolution. However, existing platforms do not contain any AI that helps to influence the outcome of a dispute. Resolution is a manual process performed by the lawyers.

To reiterate what was previously discussed, \emph{Online Dispute Resolution for Maritime Collisions} aims to deliver an abstract ODR platform whereby a module containing maritime law business logic can be plugged into the system. The module itself should ask maritime-law-specific, structured questions, interpret the answers by both parties and suggest a resolution according to maritime law.

Although it is the maritime collision AI that has never been attempted in ODR before - thus breaking new ground - the most valuable deliverable in this project would be the core ODR platform itself for being something abstract and extensible enough to support such a module. As such, the greatest development emphasis has been put on the ODR platform, rather than the module.

\subsection{Existing ODR platforms}

@TODO

\subsection{Maritime law}

@TODO - discuss all the maritime collision background reading, clarifying requirements, etc.

The Convention for the Unification of Certain Rules of Law with respect to Collisions between Vessels [REF] describes relatively straightforward, condensed maritime law which can be translated to code. On Dr Constantina Sampani's advice, this was used as the basis for the business logic in the module. %http://www.austlii.edu.au/cgi-bin/sinodisp/au/other/dfat/treaties/1930/14.html

\section{Objectives}

These were the objectives clarified early on in the project, to be tackled incrementally:

\begin{enumerate}

    \item \textbf{To build an Online Dispute Resolution platform}. This is the minimum viable product, and is easily substantial enough to be a Major Project on its own, encompassing front-end development, back-end business logic and database integration.
    
    \item For this Online Dispute Resolution to be \textbf{tailored to Maritime Collisions}, providing some sort of conclusion given the details of the dispute. As a worst-case scenario, this may mean hard-coding business logic, rules, database schemas, and so on, to fit with maritime collision properties.

    \item For this Online Dispute Resolution to be \textbf{an abstraction, able to take a module of business logic} (perhaps Maritime Collisions, perhaps something else – hence the abstraction). The aim was to bypass step 2 altogether to arrive at this stage, as it is important to keep the system abstract.

    \item As an additional feature, the system should be able to \textbf{retrieve the most similar historical cases}, which should be of use to the lawyers involved. This will likely involve a large degree of setup work, including sourcing the cases and representing them in a consistent data structure.

    \item Following on from the ability to retrieve similar cases is the ability to \textbf{feed the details of the similar cases \emph{into the current dispute}}, thereby influencing the ``court simulation" and making this feature even more accurate and valuable.

\end{enumerate}