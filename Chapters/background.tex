\chapter{Background \& Objectives}

%This section should discuss your preparation for the project, including background reading, your analysis of the problem and the process or method you have followed to help structure your work.  It is likely that you will reuse part of your outline project specification, but at this point in the project you should have more to talk about. 

\section{Introduction} % What was your background preparation for the project? What similar systems did you assess? What was your motivation and interest in this project?

\emph{Online Dispute Resolution for Maritime Collisions} was an idea put forward by Dr Constantina Sampani, a Lecturer in Law at Aberystwyth University, and presented as a Major Project idea through Dr Alexandros Giagkos, a Research Associate in the Computer Science department at the same university. [REF] % https://www.aber.ac.uk/en/law-criminology/staffdirectory/cos21/
% http://www.aber.ac.uk/en/cs/staff-list/staff_profiles/?login=alg25

The idea was to create an Online Dispute Resolution (ODR) platform that is able to suggest a resolution using artificial intelligence (AI). Maritime law was considered to be a good example to demonstrate this, as it was thought to be quite simple, concise and relatively straightforward to translate into code. It was hoped that, with the foundation work laid out in the maritime law business logic, interpretation of more complex laws could be automated in future.

With this in mind, as well delivering the maritime collision AI, it was important to create an abstract ODR platform that could take any arbitrary module of business logic other than just maritime law, so that it could be utilised in increasingly sophisticated ways in future iterations of the project.

\section{Background \& Analysis} % Taking into account the problem and what you learned from the background work, what was your analysis of the problem? How did your analysis help to decompose the problem into the main tasks that you would undertake? Were there alternative approaches? Why did you choose one approach compared to the alternatives? There should be a clear statement of the objectives of the work, which you will evaluate at the end of the work. In most cases, the agreed objectives or requirements will be the result of a compromise between what would ideally have been produced and what was felt to be possible in the time available. A discussion of the process of arriving at the final list is usually appropriate.

\subsection{An introduction to ODR}

ODR is a specialised type of Alternative Dispute Resolution (ADR), which refers to ``any means of settling disputes outside of the courtroom" [REF]. ADR can include negotiation, mediation, arbitration and so on. % reference = https://www.law.cornell.edu/wex/alternative_dispute_resolution

A recent article [REF] suggested that ODR could be used to settle non-criminal cases of less than £25,000 to reduce the expenses generated by taking cases to court. %http://www.bbc.co.uk/news/uk-31483099

There are various motivations for ODR. Since neither party is required to travel to a physical courtroom, disputes can be settled more quickly, conveniently and at lower cost. Modria, ``The World's Leading Online Dispute Resolution Experts" [REF], cite three general reasons to use ODR: % reference = http://modria.com/products/

\begin{itemize}
    \item Reduced legal risk - resolving disputes quickly decreases the chance of lawsuits because your customers feel like they're being heard.
    \item Lower operating costs - managing disputes online lowers travel expenses.
    \item Increase customer loyalty - fast, fair resolutions means happier customers.
\end{itemize}

Existing ODR platforms allow lawyers to open disputes on behalf of their clients, upload documents as evidence and communicate via online messaging, hopefully reaching an amicable resolution. However, in most cases, existing platforms do not contain any AI that helps to influence the outcome of a dispute: resolution is a manual process performed by the lawyers.

To reiterate what was previously discussed, \emph{Online Dispute Resolution for Maritime Collisions} aims to deliver an abstract ODR platform whereby a module containing maritime law business logic can be plugged into the system. The module itself should ask maritime-law-specific, structured questions, interpret the answers by both parties and suggest a resolution according to maritime law.

Although it is the maritime collision AI that has never been attempted in ODR before - thus breaking new ground - the most valuable deliverable in this project will be the core ODR platform itself for being something abstract and extensible enough to support such a module. As such, the greatest development emphasis has been put on the ODR platform, rather than the module.

\subsection{Existing ODR platforms}

The market leader for online dispute resolution is Modria [REF], who claim to have resolved 400,000,000 disputes [CHECK THIS AND REF]. It is a cloud-based platform built by the team that created the world's largest online resolution systems at eBay and PayPal [REF].

Modria is proprietary software as a service (SAAS), existing only on Modria's own cloud-based servers to those willing to pay for its services (hereafter referred to as `subscribers'). It is marketed towards e-commerce retailers, as this sector is closely related to Modria's historical area of expertise, but Modria claims to be able to resolve dispute of  ``any type and volume". % http://modria.com/product/

Subscribers can sign into the hosted interface to view all disputes against their organisation and also to define automated rules. For example, a customer orders a \$500 item from the subscriber's website and complains that it arrived lightly damaged. Through gathering information from the customer, Modria detects that this is a high-value customer who has purchased 3 items from the website and has had no previous disputes, so automatically offers a partial refund of \$50 according to the subscriber's resolution rules. % http://modria.com/how-it-works/

Modria is ideally suited to e-commerce companies beginning to feel the financial and logistical burden of trying to manually resolve hundreds or thousands of disputes. As I am not an e-commerce company, a request for a demo and several follow-up emails have gone unanswered, and unfortunately there is no other information available on the website. I would have liked answers to the following questions:

\begin{itemize}
\item How does the Modria platform cope with dispute types that aren't just simple e-commerce transactions?
\item Are there any plans to allow developer expansions, either through a remote API or an embedded module/plugin facility? What kind of components would Modria expect developers to create?
\item How much does a Modria subscription actually cost?
\end{itemize}

\subsubsection{Examining Modria's Resolution Console}

Subscribers decide rules through their `Resolution Console', an example of which is on Modria's website [REF]:

\begin{lstlisting}
If (Customer is Low Risk) and (Dispute Amount is less than $10) and (Customer Disputes Filed Account Lifetime is 0) then (Authorize Full Refund) and (Close Case).
\end{lstlisting}

This appears to be quite a powerful feature and allows the application of artificial intelligence to automatically resolving disputes. However, it is difficult to know the scope to which these rules can be modified.

Given the heavy emphasis on e-commerce transactions, the examples on the website only refer to customers, dispute amounts, transaction periods and so on. For resolving something like a maritime collision, the Resolution Console would require heavy configuration to be able to interpret maritime law into a set of rules:

\begin{lstlisting}
If (First Agent claims it was an Accident) and (Second Agent claims it was an Accident) and (First Agent says they were responsible) and (Second Agent says the other Agent was responsible) then (First Agent must Pay All Damages)
\end{lstlisting}

The above example shows how Modria might be able to support some interpretation of the law, though this is already becoming quite complicated. Now let's consider other factors we might want to factor into the resolution. We may want to retrieve the 3 most similar historic cases and feed them back into the resolution logic:

\begin{lstlisting}
If (Similar Case 1) and (Similar Case 2) and (Similar Case 3) are all in favour of (First Agent) then (Second Agent must Pay All Damages)
\end{lstlisting}

Again, this looks doable in theory, but how do we retrieve the most similar historic cases? Are we able to feed that into the Resolutions Console? Surely we'd need to write custom code to implement such functionality, and the Modria platform would need to support the execution of arbitrary code.

This and any other examples more complicated than low-value e-commerce disputes highlight the void that still exists in the world of ODR. What seems to be missing is an ODR platform that allows developers to create and plug in arbitrary modules of logic. These could be custom built to be ideally suited to resolving a specific kind of dispute, such as maritime collision disputes, divorce cases and slander/libel cases.

Such a platform would probably also have to be open source, since the developer needs to be able to hook into events and/or functions exposed by the underlying platform. Modria is unlikely to ever offer plugin functionality as it would not make sense for it to go open-source. In the words of open-source advocate Eric S. Raymond, ``when the rent from secret bits is higher than the return from open source, it makes economic sense to be closed-source". It is unlikely that the return from independent peer-review would be more valuable than the subscriptions to Modria's closed-source platform on account of there being no alternative option. If Modria were open-source, subscribers would have the option instead to download and compile the source code on their own servers, thereby paying Modria nothing.

\subsection{Maritime law}

Some time was spent examining different maritime law documents. Maritime law differs depending on the jurisdiction of the territories in which the collision took place, so it would be necessary for the maritime law AI implementation to take this into account. This might mean asking the parties where the collision took place and then applying the correct maritime law for that jurisdiction, or it might mean creating multiple maritime collision AI modules and having the parties manually select the correct one for their dispute.

In my research, I identified two key maritime conventions which could be implemented in the module. The first and most comprehensive convention was the Convention on the International Regulations for Preventing Collisions at Sea, 1972, split into five large sections and comprising a total of 38 overarching rules. This would be a good candidate for implementing in the module, but would require very thorough testing and perhaps quite an advanced logic mechanism, such as a neural network, due to its size and complexity. %http://www.admiraltylawguide.com/conven/collisions1972.html

The Convention for the Unification of Certain Rules of Law with respect to Collisions between Vessels [REF], on the other hand, describes relatively straightforward, condensed maritime law which can be translated to code. On the advice of Dr Constantina Sampani (hereafter referred to as `the client'), this was used as the basis for the business logic in the module. %http://www.austlii.edu.au/cgi-bin/sinodisp/au/other/dfat/treaties/1930/14.html

% for below paragraph, see http://ashton.codes/blog/wp-content/uploads/2015/02/cba1_OutlineProjectSpecification.pdf
Additionally, some time was spent gathering around 150 historical maritime collision cases from various sources [REF] [REF] [REF] for possible inclusion in the module. %http://emsa.europa.eu/marine-casualties-a-incidents/casualties-involving-ships/collision.html, accessed February 3rd, 2015. A collection of 100 case studies.
% http://www.offshoreblue.com/safety/accident-studies.php, accessed February 3rd, 2015. A collection of 19 case studies.
%http://www.admiraltylaw.com/collisions.php, accessed February 3rd, 2015. A collection of over 30 case studies.

\section{Objectives}

These were the objectives clarified early on in the project, to be tackled incrementally:

\begin{enumerate}

    \item \textbf{To build an Online Dispute Resolution platform}. This is the minimum viable product, and is easily substantial enough to be a Major Project on its own, encompassing front-end development, back-end business logic and database integration.
    
    \item For this Online Dispute Resolution to be \textbf{tailored to Maritime Collisions}, providing some sort of conclusion given the details of the dispute. As a worst-case scenario, this may mean hard-coding business logic, rules, database schemas, and so on, to fit with maritime collision properties.

    \item For this Online Dispute Resolution to be \textbf{an abstraction, able to take a module of business logic} (perhaps Maritime Collisions, perhaps something else – hence the abstraction). The aim was to bypass step 2 altogether to arrive at this stage, as it is important to keep the system abstract.

    \item As an additional feature, the system should be able to \textbf{retrieve the most similar historical cases}, which should be of use to the lawyers involved. This will likely involve a large degree of setup work, including sourcing the cases and representing them in a consistent data structure.

    \item Following on from the ability to retrieve similar cases is the ability to \textbf{feed the details of the similar cases \emph{into the current dispute}}, thereby influencing the ``court simulation" and making this feature even more accurate and valuable.

\end{enumerate}