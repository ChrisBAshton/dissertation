\chapter{Evaluation}\label{chapter:evaluation}

\section{Were the requirements met?}

As a reminder, the original requirements were identified as the following, to be tackled iteratively:

\begin{enumerate}

    \item Find or build an Online Dispute Resolution platform.

    \item Tailor the platform towards maritime collision disputes.

    \item Make the platform abstract, able to take a module of business logic.

    \item The maritime collision module should be able to retrieve the most similar historical cases.

    \item The details of these historical cases should be fed back into the details of the current dispute, thereby influencing the court simulation.

\end{enumerate}

The first three of these were the core requirements, but I distinctly remember that the fourth and fifth requirements were my own idea. The thought of making use of historical maritime collision cases excited me at the time and felt like a logical progression for the module. Both my supervisor and the customer were happy to agree to make this a part of the requirements specification.

Over the course of the project, I was able to look beyond the specialised area of maritime law. The light bulb moment came when I considered the commercial viability of the project (see appendix~\ref{appendix:commercialViability}), particularly the comparison with WordPress' commercial model. It was then that I saw the big picture and realised that the success of the platform would depend entirely on its extensibility, its ease of use, its ease of development and its branding.

Though the first three requirements remained, the fourth and fifth requirements took a back seat whilst I developed the platform surrounding SmartResolution. It could be argued that the first three of the original requirements were correctly identified and delivered, whereas the remaining two were surpassed by more pressing and relevant requirements. These were also subsequently delivered.

I wanted to create something that could be physically used in the real world at the end of my project and tried to see the big picture at all times. The big picture is a robust and fully extensible core ODR platform supporting an infinite number of modules that can be developed to fulfil an infinite number of uses.

It was important that the core platform offered enough hooks and an expansive-enough API to allow for the requirements we haven't even considered yet. It was equally important that there should be a straightforward way to install said modules into one's installation of the SmartResolution software. At this stage I created an embedded marketplace and installation wizard, taking inspiration from WordPress' admin panel.

If I had foregone the SmartResolution marketplace component entirely, I could have invested around a fortnight of extra development time in the maritime collision module, and perhaps accomplished something more exciting and groundbreaking in the AI. However, without a centralised means of browsing and installing modules, and indeed a centralised means of downloading the core platform itself, I feared for the future of the project. I like to think that, with the platform built and the documentation plentiful, developers are empowered to develop their own modules for this sector which has recently been gaining in popularity and public awareness.

In essence, the minimum requirements were a working ODR platform and a maritime collision module prototype. Anything beyond that in terms of delivery was my own prerogative, whether that was a SmartResolution Marketplace or a maritime collision module that is influenced by historical cases. In that respect, the core software and the basic maritime collision module satisfy the original needs of the customer.

To summarise, the software deliverables evolved from an extensible ODR platform and a sophisticated maritime collision module, into the following:

\begin{itemize}
\item Core ODR Platform
\item Simple maritime collision module
\item SmartResolution website and comprehensive documentation
\item SmartResolution live demo and automated deployment
\item SmartResolution `marketplace', allowing the perusal and downloading of all available modules
\end{itemize}

\section{Comparison with Modria}

At the beginning of this report, I identified Modria as the market leader for online dispute resolutions. I'd like to take this opportunity now to highlight the differences between the two and the pros and cons of each.

Modria is well suited to low-value e-commerce disputes. It provides a hosted platform which online retailers can become subscribers to, after which they can sign into their hosted area and view the disputes against their organisation. No developer knowledge is required and setup is minimal.

Subscribers are able to set resolution rules to automate the results of disputes as much as possible, thereby cutting costs and freeing up staff for other duties. In that respect, Modria has encoded some artificial intelligence into its system, which the subscriber is able to configure to their own needs and desires.

What Modria cannot offer at this stage is a platform that is capable of concepts beyond e-commerce disputes and terminology. Disputes come in all shapes and sizes, and  automated resolution suggestions for all of these can only be possible with a modular architecture supporting domain-specific code.

SmartResolution provides this platform. Unlike Modria, it is not hosted: it is open-source and requires some basic developer knowledge to install and configure, though it does come with detailed installation instructions. That being said, it is perfectly feasible that SmartResolution could offer hosted solutions, perhaps on a commercial site. WordPress does this itself: wordpress.org is the platform site and wordpress.com offers both free and premium hosted blogging solutions.

The important thing is not that SmartResolution is open-source, but that it is extensible. SmartResolution supports arbitrary modules of business logic, allowing developers to define custom dispute types and heuristically-driven resolutions.

The platform comes tightly integrated with the SmartResolution Marketplace, giving developers a public and legitimate means for distributing their modules. A large enough collection of quality modules - and a platform that supports the easy delivery of said modules - can be a catalyst for rapid growth, as demonstrated with Google's Android software, Apple's App Store, or indeed, WordPress' Plugin Directory.

To summarise, I think that simple disputes between two people and almost all disputes related to e-commerce are well suited to Modria's hosted ODR platform. Specialised disputes and any disputes that require representation through lawyers are best suited to the SmartResolution ODR platform.

\section{Time management}\label{section:timeManagement}

To recap, the Gantt chart in figure~\ref{uml:gantt} shows both the intended and the actual progress of the project, in blue and orange respectively. As you can see, though both project dimensions start off roughly the same, they become markedly different from mid-March onwards.

The most striking difference is probably the changing of the self-imposed deadline. Whereas the original plan had me working right up until the beginning of May, I realised later on in the project that the report would be a substantial effort of work and would also require time to print and bind, so aimed to be code-complete by the middle of April. This significantly reduced the number of weeks I'd planned to use for the development of the project.

Another noticeable difference is the amount of time spent on the core platform. I had intended to complete the core platform in just one month, but it ended up taking about 50\% longer than expected. I put this down to being overly ambitious with my expectations, not being able to find an open-source base to build upon, and disciplining myself to develop in a strictly TDD fashion. This is discussed in detail in the next section.

The very deliverables of the project also changed as time went on: there was a lower emphasis on the maritime collision module itself and more emphasis on the platform surrounding and supporting the module.

Finally, in the final stages of development, many of the tasks appear to overlap: this reflects the coupling between the various components of the project. You cannot create a maritime collision module without extending the underlying platform, but you don't necessarily know what is required of the underlying platform until you start developing the maritime collision module. You cannot create a SmartResolution Marketplace without having a finished module to host in that marketplace. You cannot create a maritime collision module without at least some understanding of maritime law. And so on.

The fact that the two Gantt dimensions are so different raises an important question: was a Gantt chart ever going to be compatible with an agile implementation, even if the first half of the project was plan-driven? In hindsight, perhaps it would have helped my planning if I had only used the Gantt chart to plan up until the early design stage, switching to agile sprints thereafter.

\section{Speed of progression}\label{section:speedOfProgression}

The design and development of the ODR platform was a major project in itself. As a result, I only had a few short weeks to concentrate my efforts on the maritime collision module, so treated this module more as a prototype than a finished product. In the words of Eric Raymond, what I've created is a ``plausible promise" of a maritime collision module.~\cite{eric:catB}

I had hoped to complete the core ODR platform more quickly than I actually managed it. In general, I did find that progress was slower than anticipated, and I'd like to briefly examine the reasons why, especially as I actually managed to negotiate some simpler requirements (removing formal resolution ``offers", etc) to help reach my deadlines.

It is partly down to my busy schedule: company work and administration, travel, visiting family and friends. My self-imposed deadline was always going to be an ambitious milestone to adhere to. However, going beyond just external commitments, I believe that coding in a test-driven way has slowed me down.

Having to write integration and unit tests, being informed by Travis several minutes later that I've broken the build, having to go back and fix tests, having to refactor my unit tests when I refactor my codebase - these have all factored into a more drawn-out development process. I think I probably would have finished the core platform sooner had I not disciplined myself to write tests throughout.

Of course, if I hadn't developed in a test-driven way, there's no way of knowing how much time I would have spent manually testing or fixing complicated bugs that start at one point and proliferate throughout the system. It is very possible that my development progress might have fallen further behind or even ground to a halt. Regardless of time, I'm extremely satisfied to have this collection of tests that cover every aspect of my codebase, as they allow me to refactor hundreds of lines of code and automatically validate that everything still works. The development overhead on writing tests is easily worth it for the ability to refactor without anxiety.

TDD aside, ``requirements creep" set in over the course of the project with stakeholders clarifying new requirements such as a file upload facility, the ability to view user profiles, and so on. These were in addition to my own self-imposed requirements, such as automated AWS deployment. I'd estimate that each new requirement adds at least two days to the development phase: one day to build and test, and the equivalent of a day in ongoing maintenance when refactoring the code.

Overall, I think I was just too ambitious in aiming to complete the core platform, in a fully test-driven way, to the Gantt schedule that I created. My only regret in terms of time management is that I spent too long preparing for the project, spending around 2-3 weeks clarifying requirements which have since evolved naturally anyway, as well as reading around the subject of maritime law and gathering historical cases: these ended up playing much less of a role in my project than the dissertation title might suggest.

\section{Appropriateness of the design}

\subsection{Choice of language and framework}

I believe that the choice of PHP as the implementation language was correct. SmartResolution is easily deployable as almost all servers support PHP; server support for Ruby, Node and other languages is less common.

In hindsight, the choice of F3 as a framework was an interesting one. To recap what I said in the design section: ``F3 is fundamentally different to its competitors because I could slot F3 into my code, rather than slotting my code into F3." I emphasised the need to be agile and the disadvantage of being locked into specific directory structures in large-scale frameworks such as Symphony or Zend.

Now that the project is complete, its directory structure has actually become quite complex, encompassing a rich and deep MVC separation with additional directories for database-querying middleware classes, classes that handle the module API, classes representing dispute states, and so on. F3 copes well with this, but perhaps a more heavyweight framework would have enforced additional advantages, such as namespaced classes or handling autoloading\footnote{Composer generates \lstinline{vendor/autoload.php}, but \lstinline{webapp/autoload.php} is manually created and must explicitly pull in various files and directories Frameworks such as Zend provide ways of hooking into the autoloader.}.

Although SmartResolution has developed into a project that would have benefited from the features provided by a larger framework, F3 definitely had a lower learning curve. This meant that I was able to start implementing features the day I started developing, rather than spending days or weeks getting to grips with a heavyweight framework. I don't regret my choice, especially since frameworks tend to be moving towards a modular style anyway, as discussed in subsection~\ref{subsection:framework}.

\subsection{Appropriateness of implementation}

Figure~\ref{uml:class} contains the class diagram showing the object-oriented nature of the system and the interaction between the models. I'm quite happy with the final design, but certain elements could still be improved.

For instance, the \lstinline{MediationState} class encapsulates a lot of business logic, representing the mediation state of a dispute right from mediation proposal, to choosing a mediation centre, to the mediation centre offering a list of available mediators, etc. The state pattern worked well for the dispute itself, and should maybe have been extended to incorporate this mediation logic too as it would have made state querying more consistent.

I believe that, in hindsight, more time spent on an up front design would have benefited the final design. It may also have meant I'd have side-stepped some of the difficult periods of the project, such as when I refactored all of the models to take an array of data instead of a database record ID in the constructor. I may have spotted the database/model coupling sooner and implemented the system correctly in the first place.

At the higher level, I'm very pleased with the MVCR structure of the application. This pattern made it easy to assign routes to different handlers and to separate the concerns of the model with the wider concerns of the controllers and the data-querying in the application itself.

\section{Future improvements}

I've delivered three major, fully-functional components totalling around 10,000 lines of code, but there is so much more I'd have liked to have added if time allowed for it. These improvements are separated by component below.

\subsection{SmartResolution}

The admin account must be created manually in the database. In future, it would be nice if this were a part of a user-friendly installation process complete with a nice GUI. This process would prompt the user for information such as admin email and password and which kind of database the installation should use (such as SQLite or MySQL).

Assuming an admin account has been created, when the admin logs into their account they're presented with three options: Marketplace, Modules and Customise. The latter option is a placeholder and contains no functionality. It was hoped that this might later allow the admin to customise their installation by changing the SmartResolution logo or enabling/disabling mediation. I would also like to have added the ability to switch themes, as WordPress allows bloggers to easily change the look and feel of their site by activating a new theme. This is something that is perfectly possible in SmartResolution, thanks to its MVC architecture.

\subsection{Maritime collision module}

The original requirements suggested that the module could pull in similar historical cases, firstly as a useful reference for the lawyers but secondly as a heuristic that directly affects the module's suggested resolution. Naturally, this is something I'd have liked to have added if the time was available.

The research discussed in subsection~\ref{subsection:maritimeLaw} suggested that maritime law is different depending on the geographical location of the collision. In future, my maritime collision module could be expanded to apply different laws depending on the location of the collision, or else multiple different modules could be created and the choice of which module to use made the responsibility of the agents.

\subsection{SmartResolution Marketplace}

Currently, there is nothing clever regarding putting modules on the SmartResolution Marketplace: the process of zipping up modules, deploying them to the marketplace and editing the marketplace JSON feed is manual.

This is something I would want to automate in the future, perhaps providing an upload interface which takes a GitHub repository URL and zips up the contents automatically. However, the manual process currently in place does at least give me the advantage of being able to check the code and make sure that there is nothing insidious about its contents before making it available on the marketplace.

A more exciting feature I would have liked to have added is inspired by Modria: hosted ODR. It is perfectly feasible to have a corporate ODR service hosted on smartresolution.org where corporations pay a subscription for their SmartResolution installation to be accessible at their own subdomain of the SmartResolution website.

\section{Relevance to degree scheme}

My degree scheme is \emph{Software Engineering} and I believe I've developed this project to a high standard in true software engineering style, using the following best-practice guidelines:

\begin{itemize}
\item Almost fully object-oriented, but follows a functional paradigm where appropriate.
\item Uses PDO for database interaction, to protect against SQL injection attacks.
\item Uses F3 as a framework, delegating the heavy lifting to third-party modules which handle business logic such as password encryption and HTTP routing. Software reuse is a key part of software engineering.
\item Uses Bootstrap at the front-end, to minimise the wastage involved in duplicating well-established UI design and to enable mobile responsiveness by default.
\item Clean, semantic HTML markup, W3C-validated and friendly to screen readers.
\item Appropriate use of numerous well-established design patterns, including the MVCR composite design pattern for separation of concerns, state pattern for encapsulation of business logic and publish-subscribe pattern for decoupling the core platform with any installed modules.
\item Javadoc-style documented comments throughout codebase.
\end{itemize}

At a higher level, I've utilised industry-standard tools and best practices to ensure my code stays at the highest quality:

\begin{itemize}
\item The entire SmartResolution core platform is covered by low-level unit tests and high-level Cucumber tests, testing all aspects of functionality. If any functionality breaks as the result of an ill-thought-out refactor, at least one test should fail.
\item Dependency management through Composer and RubyGems, both monitored remotely through the Gemnasium dependency monitoring service.
\item Travis: a continuous integration platform configured to automatically run all of my project's tests whenever a commit is pushed to the repository. This way, even if I forget to run the test suite, the tests will still be run and I'll be warned. The build status is also automatically pulled into smartresolution.org and the SmartResolution repository README, so anybody can see at a glance whether or not the latest version is stable.
\item CodeClimate: an automated code review service which analyses the quality, style and security of every line of code, giving a helpful second opinion. The suggestions from this service were directly fed back into the design of my code.
\end{itemize}

At the highest level, I've used services to keep on top of my project management:

\begin{itemize}
\item GitHub - a version management system allowing me to develop branches and then merge into the master branch following a line-by-line code-review. GitHub also provides a place to log issues including bugs, possible enhancements, and so on. I used these in conjunction with GitHub's `milestone' feature, which allows me to group issues into related milestones.
\item JIRA - I used my own installation of JIRA to plan out the project management aspects of the major project, including planning for the mid-project demonstration.
\item Gantt chart - an industry-standard technique for planning and monitoring progress.
\item PHPDoc - a tool to generate HTML documentation from my DocBlock comments, which has been a useful reference when refactoring.
\end{itemize}

\section{Summary}

SmartResolution has grown into quite a substantial platform which has scaled well to the challenges of a large codebase by being engineered in a disciplined and sustainable way.

I believe that the platform developed in this project fulfils a real need and has benefited from taking inspiration from WordPress for its API design and its marketing strategy. Though online dispute resolution doesn't have the wider appeal that blogging might have, SmartResolution could easily form the basis of a successful ODR provider's platform in future and become something of a household name.

More than just offering a feature-complete base to build upon, SmartResolution supports modular extensibility for any features that it does \emph{not} have. The core software would probably require further modification to support the events and global functions that were not required by the maritime collision module - it is impossible to know in advance what kind of API might be required by the developers of future modules - but the core infrastructure is now in place.

The maritime collision module implementation is simple but well designed. It demonstrates what SmartResolution's modular build makes possible, providing a plausible promise of what the module could be if it were given a little more attention. To implement everything I originally wanted to implement would be another major project in itself and is something worth considering for one of next year's students.

Finally, I feel that the development of the SmartResolution website and the accompanying developer documentation has solidified a collection of fragmented concepts in a tangible way. Building the platform website and encouraging developer activity through providing a marketplace has steered this ODR platform in a direction where it may otherwise have been lost at sea, if you can pardon the pun.

10,000 lines of code, 20,000 words and 400 hours of effort later, and the project is ready to be deployed to production, used by people, and further enhanced according to their feedback. I was recently able to demo the final version of the software with the customer and they were very pleased. Dr. Constantina Sampani will now be presenting what we've built at the Maritime Arbitrators' conference in Hong Kong in May 2015.

There has already been some discussion regarding taking this project further, perhaps as a major project for one of next year's final-year students to make the maritime collision module more sophisticated. It will be really interesting to see what happens to SmartResolution in this field which is rapidly gaining significance and acceptance in the eyes of the law. I hope I've created a solid foundation to build upon.