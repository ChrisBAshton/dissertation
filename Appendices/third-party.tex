\chapter{Third-Party Code and Libraries}

I will begin by listing the third-party code, libraries, frameworks and images used by the three components of this dissertation. The full description for each dependency can be found at the end of this appendix, in alphabetical order.

\section{SmartResolution website}

smartresolution.org uses a number of third-party frameworks and services:

\begin{itemize}
\item Bootstrap: this is used for the front-end styling and as a UI framework.
\item Fat-Free Framework: this powers the back-end, allowing me to define HTTP routes and their handlers.
\item phpDocumentor: this is used to generate API documentation from the Docblock comments in the core SmartResolution software.
\item SlickNav: responsive mobile menu.
\item jQuery: a dependency for the plugin above.
\end{itemize}

In addition to the SmartResolution website, I developed a demo video showing the workings of the SmartResolution software and the maritime collision module; this video is embedded on the homepage of the SmartResolution website. This video is hosted on YouTube and makes use of YouTube's annotation facility to give useful commentary throughout the video. I overlayed a music track by Eric Matyas, which is free under the Creative Commons by Attribution license\footnote{Licensing is discussed on the provider's website at \url{http://soundimage.org/}}. The track is entitled `GAME MENU\_v001' and is available at \url{http://soundimage.org/looping-music/}

The design and creation of the installation icons\footnote{Available at \url{http://smartresolution.org/documentation}} were outsourced to a friend of mine, Lani Cossins. The workflow images on the website\footnote{Available at \url{http://smartresolution.org/workflow}} and in appendix~\ref{appendix:workflow} were my own design, but were later digitised by Rosie Bettles. 

It is worth explaining that, though smartresolution.org is hosted on AWS, it is server-agnostic and no AWS-specific assets exist in the repository. Therefore, I will not be discussing the use of AWS here.

\section{SmartResolution}

As SmartResolution is more substantial than the website which hosts it, it is necessary to split its dependencies into certain categories.

\subsection{Front-end}

\begin{itemize}
\item Bootstrap: for front-end styling and as a UI framework.
\item jQuery Date and Time picker: for a user-friendly way of negotiating dispute lifespans.
\item jQuery: a dependency required by the plugin above.
\item The dashboard icons used in the core platform are designed by Freepik.
\end{itemize}

\subsection{Back-end}

\begin{itemize}
\item Fat-Free Framework: routes HTTP requests to controllers and also handles database interaction and password encryption.
\item SQLite: the RDBMS used by SmartResolution for persistence.
\end{itemize}

\subsection{Development dependencies}

\begin{itemize}
\item Rubygems: for handling Ruby dependencies.
\item Composer: for handling PHP dependencies.
\item Cucumber: the BDD framework in use for end-to-end testing.
\item RSpec, Capybara and Poltergeist as the Cucumber drivers.
\item PHPUnit: for providing methods for (and subsequently executing) PHP unit tests.
\item Symfony's YAML component: for handling PHP's parsing of YAML files, used by SmartResolution's PHP unit tests.
\end{itemize}

\section{Maritime collision module}

No third-party libraries were used in the development of the module. All code is native PHP/HTML and all assets (such as the anchor image) were taken from the public domain (in this case, pixabay.org).

\section{Descriptions}

All of the following were used without modification.

\textbf{Bootstrap} (v3.2.2) - MIT - \url{http://getbootstrap.com/}

Bootstrap defines sensible browser defaults (instantly improving unstyled webpages), a responsive 12-column layout and all manner of useful component classes. These classes help represent standard program `states' such as success, error, new information, and so on. ``Bootstrap is the most popular HTML, CSS, and JS framework for developing responsive, mobile first projects on the web.". 

\textbf{Capybara} (v2.4.4) - MIT - \url{https://github.com/jnicklas/capybara}

Used in conjunction with Ruby and Cucumber, ``Capybara helps you test web applications by simulating how a real user would interact with your app. It is agnostic about the driver running your tests"; in our case, we use Poltergeist. 

\textbf{Cucumber} (v2.0.0) - MIT - \url{https://github.com/cucumber/cucumber}

``Cucumber is a tool for running automated tests written in plain language." Cucumber features are the way I textualised my requirements at the beginning of the project. They're incredibly useful as they are written in plain language yet are linked to tests, so serve as tests and documentation all at once.

\textbf{Dashboard icons} - Freepik license - \url{http://www.flaticon.com/packs/web-pictograms}

These icons are free for use but the Freepik license requires that the phrase ``designed by Freepik" is displayed on the webpage that uses the icons. I have fulfilled this by putting an attribution to them in the footer of the SmartResolution software. License URL: \url{http://cdn.flaticon.com/license/license.pdf}

\textbf{Fat-Free Framework} (v3.4) - GPLv3 - \url{https://github.com/bcosca/fatfree}

F3 provides a solid foundation including ``an easy-to-use Web development tool kit, a high-performance URL routing and cache engine, built-in code highlighting, and support for multilingual applications." I particularly found the database interaction and routing modules useful.

\textbf{jQuery} (v2.1.3) - MIT - \url{https://jquery.org/}

jQuery is the most popular JavaScript library in the world, providing useful methods on DOM elements in a cross-browser and backwards-compatible way. Many JavaScript libraries and plugins require jQuery as a dependency, such is its ubiquity. License URL: \url{https://jquery.org/license/}

\textbf{jQuery Date and Time picker (plugin for jQuery)} (v2.4.1) - MIT - \url{https://plugins.jquery.com/datetimepicker/}

A user-friendly date and time picker, used by the SmartResolution core software to make it easier for agents to propose start and end dates for their disputes.

\textbf{phpDocumentor} (v2) - MIT - \url{https://github.com/phpDocumentor/phpDocumentor2}

Having marked up SmartResolution with detailed, semantic comments, I needed a program to parse those comments and generate API documentation in HTML. phpDocumentor was exactly what I needed.

\textbf{PHPUnit} (v4.6) - BSD 3-Clause - \url{https://github.com/sebastianbergmann/phpunit/}

``PHPUnit is a programmer-oriented testing framework for PHP". I use PHPUnit to unit-test the SmartResolution core software. BSD license information: \url{http://opensource.org/licenses/BSD-3-Clause}

\textbf{Poltergeist} (v1.6.0) - MIT - \url{https://github.com/teampoltergeist/poltergeist}

Poltergeist is a driver for Capybara which allows one to run Capybara tests on a headless WebKit browser (provided by PhantomJS). I chose it because it is quick to run and supports JavaScript, which is required by some SmartResolution features.

\textbf{RSpec} (v3.2.0) - MIT - \url{https://github.com/rspec/rspec/}

RSpec is used for defining assertions, exposing methods such as \lstinline{assert(some_condition)}. It provides the very foundation of my Cucumber feature tests.

\textbf{Rubygems} (v2.2.2) - MIT - \url{https://github.com/rubygems/rubygems}

RubyGems is a package management framework for Ruby. It can be used to automatically download all of SmartResolution's Ruby dependencies, including Cucumber, Capybara and RSpec.

\textbf{SlickNav} (v1.0.3) - MIT - \url{http://slicknav.com/}

Duplicates a given menu and styles it in a mobile-friendly (i.e. dropdown) style. The application using it simply has to add a media query to their CSS to hide the `mobile' menu and show the `desktop' menu when an arbitrary breakpoint is reached.

\textbf{SQLite} (v3.7.13) - Public domain - \url{http://www.sqlite.org/}

A simple, fast relational database management system. It is important to note that this choice of RDBMS is not particularly important, as another RDBMS can be swapped into SmartResolution with relative ease. Copyright information: \url{http://www.sqlite.org/copyright.html}

\textbf{Symfony's YAML component} (v2.6) - MIT - \url{https://github.com/symfony/Yaml}

Parses a string containing YAML-formatted data. YAML (a recursive acronym, standing for ``YAML Ain't Markup Language") is a terse way of representing structured data, and has gained popularity for use with languages such as Ruby and Python. PHP has no built-in YAML parser, so I used Symfony's YAML component, which is an industry-standard YAML parser for PHP.