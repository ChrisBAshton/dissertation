\chapter{Requirements Gathering}\label{appendix:requirements}

My plan was always to use behaviour-driven development in my project, and I believe passionately in keeping all documentation as closely in line with the code as possible. I used Gherkin syntax to describe the features, and linked these up to an executable Ruby and Cucumber test suite.

These are the original requirements, agreed and signed off by all of the project's stakeholders. % https://github.com/ChrisBAshton/smartresolution/tree/bac098fc049ab0654e242bc43040d7e11cf6c2c4/features

\section{Account Creation}

\begin{lstlisting}
Feature: Account Creation
    I should be able to register an account of a certain type, e.g. Company/Agent
    And I should be able to log into said account

  Scenario: Company registration
    Given I am an authorised representative of the Company
    When I attempt to create a new Company account
    Then the account should be created
    # @TODO - as discussed in the supervisor meeting, we could add an admin verification stage at this stage.
    # This could be as complicated as we want to make it, so for now, let's add a boolean in the database that
    # says if the Company is verified or not. Make it verified by default, but carry out isVerified checks at login.
    # We can add the verification steps later and make Companies unverified by default.
    # At that stage, we need to add additional features, e.g. Given I am unverified, When I try to log in, Then I should
    # Not be allowed to do anything.

  Scenario: Agent initiates Case against a Company not registered to the system
    Given I have not yet registered a Company account
    And a Dispute has been initiated against my Company
    When I attempt to create a new Company account
    Then the account should be created
    And the my Company should be automatically linked to the dispute

  Scenario: Company login with valid credentials
    Given I have registered a Company account
    When I attempt to log in with valid credentials
    Then I should be logged into the system

  Scenario: Company login with invalid credentials
    Given I have registered a Company account
    When I attempt to log in with invalid credentials
    Then an authentication error should be displayed
    # @TODO - as discussed, correct and incorrect login attempts should be logged so that we can later add
    # additional security such as locking out accounts after a threshold of unsuccessful attempts is reached.

  Scenario: Agent login
  Scenario: Mediator login
  Scenario: Mediation Centre login

  Scenario: Create Agent account
    Given I have logged into a Company account
    Then I should be able to create an Agent account
    And the Agent should be sent an email notifying them they've been registered

  # Exactly the same process as for Company registration. There should be a drop-down list that lets registering users
  # select whether they're registering as a Company or a Mediation Centre
  Scenario: Mediation Centre registration
  Scenario: Create Mediator account
\end{lstlisting}

\section{Dispute Creation}

\begin{lstlisting}
Feature: Dispute creation
    Given I am logged into an authorised account
    Then I should be able to create a Dispute
    
  Scenario: Creating a Dispute
    Given I am logged into a Company account
    Then I should be able to create a new Dispute

  Scenario: Allocating a Dispute to an Agent
    Given I have created a Dispute
    And I have created an Agent
    Then I should be able to allocate the Agent to the Dispute # this should also be possible AT the Dispute creation stage

  Scenario: Submitting a Dispute
    Given a Dispute has been assigned to me # by my Company, regardless of who instigated the Dispute
    When I write a Dispute summary
    And I choose to submit the Dispute to the system
    Then the Dispute should be submitted

  Scenario: Initiating a Dispute against a Company
    Given I have submitted a Dispute
    Then I should be able to initiate it against another Company
    # We pick from a drop-down list of Companies in the system
    # or provide a Company email address inviting them to register.

  Scenario: Being initiated a Dispute
    Given a Dispute has been initiated against my Company
    And I have created an Agent
    Then I should be able to allocate the Agent to the Dispute
\end{lstlisting}

\section{Dispute}

\begin{lstlisting}
Feature: Dispute (pre-Mediation)
    The Dispute is underway, both Agents are free to communicate with one another,
    propose offers, attach evidence, etc.
  
  Background:
    Given the Dispute is fully underway
    And the Dispute is not in Mediation

  Scenario: Free communication
    Then I should be able to communicate with the other Agent freely

  # The "Propose Resolution" mechanism outlined below is a separate facility to above.
  # Think of the free communication as a private messaging system (which gets blocked when
  # we go into Mediation, then re-opened with the additional Mediator person when entering
  # round-table communication).
  # The offer mechanism is a more formalised communication, where you offer a certain amount,
  # under X conditions - Accept | Deny | Propose Counter Offer

  Scenario: Make an offer
    Then I should be able to send the other Agent an offer

  Scenario: Accept the offer
    Given I have been sent an offer
    Then I should be able to accept the offer
    And the Dispute should close successfully

  Scenario: Propose counter-offer
    Given I have been sent an offer
    Then I should be able to propose a different offer

  # Note:
  # I can also Decline an offer by taking the Case to Court - see dispute_independent.feature "Take the Dispute to Court".
  # I can also propose Mediation. See dispute_independent.feature "Start the Mediation Process"
\end{lstlisting}

\section{Dispute-Independent Features}

\begin{lstlisting}
Feature: Processes relevant to the Dispute but that are not dependent on the current state of the Dispute
    There are various functionalities that do not depend on the current state of the Dispute
    And should be accessible at any point in the Dispute

  Scenario: Start the Mediation process
    Given the Mediation process has not begun
    Then I should be able to start the Mediation process

  Scenario: Take the Dispute to Court
    Given the Dispute has not yet been resolved
    Then I should be able to Take the Dispute to Court
    And the Dispute should close unsuccessfully
\end{lstlisting}

\section{Dispute Lifespan Negotiation}

\begin{lstlisting}
Feature: Negotiating a Dispute lifespan
    When a Dispute is opened and each Company has allocated an Agent
    The Agents need to negotiate a Dispute lifespan
    i.e. the maximum length of time the Dispute can continue without resolution 
    before being automatically taken to Court.

  Scenario: Creating a Dispute lifespan offer
    Given both Agents have submitted the Dispute
    Then I should be able to make a lifespan offer # regardless of who submitted the Dispute first

  Scenario: Accepting a Dispute lifespan offer
    Given the other Agent has sent me a Dispute lifespan offer
    Then I should be able to accept the offer
    And the Dispute should start

  Scenario: Create a counter Dispute lifespan offer
    Given the other Agent has sent me a Dispute lifespan offer
    Then I should be able to make a lifespan offer
    And therefore decline their original offer

  Scenario: Renegotiating the Dispute lifespan mid-Dispute
    Given the Dispute is fully underway
    Then I should be able to make a lifespan offer
    And the Dispute should continue normally despite the renegotiation offer
\end{lstlisting}

\section{Putting a Dispute into Mediation}

\begin{lstlisting}
Feature: Mediation
    At any point in a confirmed Dispute
    Either Agent can propose Mediation
    Whereby a Mediator is introduced to help to resolve the Dispute
    If complications arise during the Mediation Creation process - e.g. a
    list of Mediators being provided to the Agents but are not suitable,
    then either Agent can restart the Mediation process.

  Background:
    Given the Dispute is underway and a lifespan has been agreed

  Scenario: Choosing a Mediation Centre
    Given both Agents have agreed to start the Mediation process
    Then I should be able to select the Mediation Centres I'm happy with
    # We do this by choosing the Mediation Centres we want AND an order of preference.

  Scenario: No mutually chosen Mediation Centres
    Given both Agents have selected the Mediation Centres they want
    And there are no matches in their choices
    Then both Agents should have the opportunity to choose again

  Scenario: One mutually chosen Mediation Centre
    Given both Agents have selected the Mediation Centres they want
    And there is only one match in their choices
    Then that should be the chosen Mediation Centre

  Scenario: Multiple mutually chosen Mediation Centres
    Given both Agents have selected the Mediation Centres they want
    And there are several matching choices
    Then one Mediation Centre must be chosen upon by both Agents
    # It's been suggested we could make Agents choose (before this step)
    # the order of preference for the Mediators, then the system could suggest
    # a Mediator based on a points system.

  Scenario: Mediation Centre is notified of the Agents' decision
    Given my Mediation Centre has been chosen by both Agents of a Dispute
    Then I should be notified that my Mediation Centre has been chosen
    And I should have the facility to offer a list of Mediators to the Agents

  Scenario: Mediation Centre provides list of Mediators
    Given a Mediation Centre has provided a list of available Mediators
    Then I should be able to view the details of each Mediator # including CV, etc
    And I should be able to select the Mediators I'm happy with

  Scenario: No mutually chosen Mediators
    Given both Agents have selected the Mediators they want
    And there are no matches in their choices
    Then both Agents should have the opportunity to choose again

  Scenario: One mutually chosen Mediator
    Given both Agents have selected the Mediators they want
    And there is only one match in their choices
    Then that should be the chosen Mediator

  Scenario: Multiple mutually chosen Mediators
    Given both Agents have selected the Mediators they want
    And there are several matching choices
    Then one Mediator must be chosen upon by both Agents

  Scenario: Mediator is notified of the Agents' decision
    Given I am a Mediator
    And I have been chosen by both Agents of a Dispute
    Then I should be notified that I have been chosen
    And I should be made to sign a confidentiality agreement

  Scenario: Mediator signs confidentiality agreement
    Given I am a mutually-chosen Mediator for a given Dispute
    And I sign the confidentiality agreement
    Then the Dispute should now be in Mediation Mode
\end{lstlisting}

\section{Dispute in Mediation}

\begin{lstlisting}
Feature: Dispute (under Mediation)
    The rules of the Dispute have now changed. All communication must be done through the Mediator.
    It is at this point that the business logic specific evidence-gathering can be applied, so that
    the artifical intelligence in the module can provide a second opinion to the Mediator.
    The Mediator, being a specialised and trained individual, can choose to ignore or amend the given
    advice.
  
  Background:
    Given the Dispute is fully underway
    And the Dispute is in Mediation

  Scenario: Block Agent A and B from communicating with one another
    Given we have not activated round-table communication
    Then I should not be able to communicate with the other Agent

  Scenario: Mediator requires further information
    Given a Dispute type was selected at the beginning of the Dispute # e.g. "maritime collision"
    Then the type-specific module should offer custom forms to the Agents to fill in

  # business logic specific stuff relating to Maritime Collisions etc MUST be put into a separate feature file
  # (in the plugin directory). This set of features must be as abstract and generic as possible.

  Scenario: Filling in the type-specific forms
    Given I am an Agent
    And I have filled in the forms provided by the type-specific module
    Then there should be no more forms to fill in
    And I should see that the system is awaiting a response from the other Agent

  Scenario: Filling in the type-specific forms as the second Agent
    Given I am an Agent
    And I have filled in the forms provided by the type-specific module
    And the other Agent has also filled in the forms
    Then there should be no more forms to fill in
    And I should see that the system is awaiting a response from the Mediator

  Scenario: AI logic is applied
    Given I am a Mediator
    And both Agents have completed the type-specific module forms
    Then I should see the results of the AI in the type-specific module
    #And I should be able to advise each Agent individually
    # Commented out the above line because it isn't testable. Essentially, the Mediator can send a
    # private message to either Agent, negotiating a resolution. It is up to the Agents to formally
    # send an offer through the "Propose Resolution" facility.

  Scenario: Accepting the Mediator's offer
    Given the Mediator has given me an offer
    Then I should be able to accept the offer
    And the Dispute should close successfully

  Scenario: Declining the Mediator's offer
    Given the Mediator has given me an offer
    Then I should be able to decline the offer
    And the Dispute should remain open

  Scenario: Sending an offer for round-table communication
    Given I am a Mediator
    Then I should be able to offer round-table communication
    # The Mediator should (through a dedicated facility) be able to propose round-table negotation,
    # whereby the free communication of all parties is enabled.

  Scenario: Accepting the offer for round-table communication
    Given the Mediator has suggested round-table communication
    Then I should be able to accept the offer
    And the Dispute should go into Round Table Mediation mode

  Scenario: Declining the offer for round-table communication
    Given the Mediator has suggested round-table communication
    Then I should be able to decline the offer
    And the Dispute should remain open and under Mediation
\end{lstlisting}
