\chapter{Requirements Gathering}\label{appendix:requirements}

My plan was always to use behaviour-driven development in my project, and I believe passionately in keeping all documentation as closely in line with the code as possible. I used gherkin syntax to describe the features, and linked these up to an executable Ruby and Cucumber test suite.

\section{Original Requirements}

These are the original requirements, agreed and signed off by all of the project's stakeholders. % https://github.com/ChrisBAshton/smartresolution/tree/bac098fc049ab0654e242bc43040d7e11cf6c2c4/features

\subsection{Account Creation}

\begin{lstlisting}
Feature: Account Creation
    I should be able to register an account of a certain type, e.g. Company/Agent
    And I should be able to log into said account

  Scenario: Company registration
    Given I am an authorised representative of the Company
    When I attempt to create a new Company account
    Then the account should be created
    # @TODO - as discussed in the supervisor meeting, we could add an admin verification stage at this stage.
    # This could be as complicated as we want to make it, so for now, let's add a boolean in the database that
    # says if the Company is verified or not. Make it verified by default, but carry out isVerified checks at login.
    # We can add the verification steps later and make Companies unverified by default.
    # At that stage, we need to add additional features, e.g. Given I am unverified, When I try to log in, Then I shoul
    # Not be allowed to do anything.

  Scenario: Agent initiates Case against a Company not registered to the system
    Given I have not yet registered a Company account
    And a Dispute has been initiated against my Company
    When I attempt to create a new Company account
    Then the account should be created
    And the my Company should be automatically linked to the dispute

  Scenario: Company login with valid credentials
    Given I have registered a Company account
    When I attempt to log in with valid credentials
    Then I should be logged into the system

  Scenario: Company login with invalid credentials
    Given I have registered a Company account
    When I attempt to log in with invalid credentials
    Then an authentication error should be displayed
    # @TODO - as discussed, correct and incorrect login attempts should be logged so that we can later add
    # additional security such as locking out accounts after a threshold of unsuccessful attempts is reached.

  Scenario: Agent login
  Scenario: Mediator login
  Scenario: Mediation Centre login

  Scenario: Create Agent account
    Given I have logged into a Company account
    Then I should be able to create an Agent account
    And the Agent should be sent an email notifying them they've been registered

  # Exactly the same process as for Company registration. There should be a drop-down list that lets registering users
  # select whether they're registering as a Company or a Mediation Centre
  Scenario: Mediation Centre registration
  Scenario: Create Mediator account
\end{lstlisting}

\subsection{Dispute Creation}

\begin{lstlisting}
Feature: Dispute creation
    Given I am logged into an authorised account
    Then I should be able to create a Dispute
    
  Scenario: Creating a Dispute
    Given I am logged into a Company account
    Then I should be able to create a new Dispute

  Scenario: Allocating a Dispute to an Agent
    Given I have created a Dispute
    And I have created an Agent
    Then I should be able to allocate the Agent to the Dispute # this should also be possible AT the Dispute creation stage

  Scenario: Submitting a Dispute
    Given a Dispute has been assigned to me # by my Company, regardless of who instigated the Dispute
    When I write a Dispute summary
    And I choose to submit the Dispute to the system
    Then the Dispute should be submitted

  Scenario: Initiating a Dispute against a Company
    Given I have submitted a Dispute
    Then I should be able to initiate it against another Company
    # We pick from a drop-down list of Companies in the system
    # or provide a Company email address inviting them to register.

  Scenario: Being initiated a Dispute
    Given a Dispute has been initiated against my Company
    And I have created an Agent
    Then I should be able to allocate the Agent to the Dispute
\end{lstlisting}

\subsection{Dispute}

\begin{lstlisting}
Feature: Dispute (pre-Mediation)
    The Dispute is underway, both Agents are free to communicate with one another,
    propose offers, attach evidence, etc.
  
  Background:
    Given the Dispute is fully underway
    And the Dispute is not in Mediation

  Scenario: Free communication
    Then I should be able to communicate with the other Agent freely

  # The "Propose Resolution" mechanism outlined below is a separate facility to above.
  # Think of the free communication as a private messaging system (which gets blocked when
  # we go into Mediation, then re-opened with the additional Mediator person when entering
  # round-table communication).
  # The offer mechanism is a more formalised communication, where you offer a certain amount,
  # under X conditions - Accept | Deny | Propose Counter Offer

  Scenario: Make an offer
    Then I should be able to send the other Agent an offer

  Scenario: Accept the offer
    Given I have been sent an offer
    Then I should be able to accept the offer
    And the Dispute should close successfully

  Scenario: Propose counter-offer
    Given I have been sent an offer
    Then I should be able to propose a different offer

  # Note:
  # I can also Decline an offer by taking the Case to Court - see dispute_independent.feature "Take the Dispute to Court".
  # I can also propose Mediation. See dispute_independent.feature "Start the Mediation Process"
\end{lstlisting}

\subsection{Dispute-Independent Features}

\begin{lstlisting}
Feature: Processes relevant to the Dispute but that are not dependent on the current state of the Dispute
    There are various functionalities that do not depend on the current state of the Dispute
    And should be accessible at any point in the Dispute

  Scenario: Start the Mediation process
    Given the Mediation process has not begun
    Then I should be able to start the Mediation process

  Scenario: Take the Dispute to Court
    Given the Dispute has not yet been resolved
    Then I should be able to Take the Dispute to Court
    And the Dispute should close unsuccessfully
\end{lstlisting}

\subsection{Dispute Lifespan Negotiation}

\begin{lstlisting}
Feature: Negotiating a Dispute lifespan
    When a Dispute is opened and each Company has allocated an Agent
    The Agents need to negotiate a Dispute lifespan
    i.e. the maximum length of time the Dispute can continue without resolution 
    before being automatically taken to Court.

  Scenario: Creating a Dispute lifespan offer
    Given both Agents have submitted the Dispute
    Then I should be able to make a lifespan offer # regardless of who submitted the Dispute first

  Scenario: Accepting a Dispute lifespan offer
    Given the other Agent has sent me a Dispute lifespan offer
    Then I should be able to accept the offer
    And the Dispute should start

  Scenario: Create a counter Dispute lifespan offer
    Given the other Agent has sent me a Dispute lifespan offer
    Then I should be able to make a lifespan offer
    And therefore decline their original offer

  Scenario: Renegotiating the Dispute lifespan mid-Dispute
    Given the Dispute is fully underway
    Then I should be able to make a lifespan offer
    And the Dispute should continue normally despite the renegotiation offer
\end{lstlisting}

\subsection{Putting a Dispute into Mediation}

\begin{lstlisting}
Feature: Mediation
    At any point in a confirmed Dispute
    Either Agent can propose Mediation
    Whereby a Mediator is introduced to help to resolve the Dispute
    If complications arise during the Mediation Creation process - e.g. a
    list of Mediators being provided to the Agents but are not suitable,
    then either Agent can restart the Mediation process.

  Background:
    Given the Dispute is underway and a lifespan has been agreed

  Scenario: Choosing a Mediation Centre
    Given both Agents have agreed to start the Mediation process
    Then I should be able to select the Mediation Centres I'm happy with
    # We do this by choosing the Mediation Centres we want AND an order of preference.

  Scenario: No mutually chosen Mediation Centres
    Given both Agents have selected the Mediation Centres they want
    And there are no matches in their choices
    Then both Agents should have the opportunity to choose again

  Scenario: One mutually chosen Mediation Centre
    Given both Agents have selected the Mediation Centres they want
    And there is only one match in their choices
    Then that should be the chosen Mediation Centre

  Scenario: Multiple mutually chosen Mediation Centres
    Given both Agents have selected the Mediation Centres they want
    And there are several matching choices
    Then one Mediation Centre must be chosen upon by both Agents
    # It's been suggested we could make Agents choose (before this step)
    # the order of preference for the Mediators, then the system could suggest
    # a Mediator based on a points system.

  Scenario: Mediation Centre is notified of the Agents' decision
    Given my Mediation Centre has been chosen by both Agents of a Dispute
    Then I should be notified that my Mediation Centre has been chosen
    And I should have the facility to offer a list of Mediators to the Agents

  Scenario: Mediation Centre provides list of Mediators
    Given a Mediation Centre has provided a list of available Mediators
    Then I should be able to view the details of each Mediator # including CV, etc
    And I should be able to select the Mediators I'm happy with

  Scenario: No mutually chosen Mediators
    Given both Agents have selected the Mediators they want
    And there are no matches in their choices
    Then both Agents should have the opportunity to choose again

  Scenario: One mutually chosen Mediator
    Given both Agents have selected the Mediators they want
    And there is only one match in their choices
    Then that should be the chosen Mediator

  Scenario: Multiple mutually chosen Mediators
    Given both Agents have selected the Mediators they want
    And there are several matching choices
    Then one Mediator must be chosen upon by both Agents

  Scenario: Mediator is notified of the Agents' decision
    Given I am a Mediator
    And I have been chosen by both Agents of a Dispute
    Then I should be notified that I have been chosen
    And I should be made to sign a confidentiality agreement

  Scenario: Mediator signs confidentiality agreement
    Given I am a mutually-chosen Mediator for a given Dispute
    And I sign the confidentiality agreement
    Then the Dispute should now be in Mediation Mode
\end{lstlisting}

\subsection{Dispute in Mediation}

\begin{lstlisting}
Feature: Dispute (under Mediation)
    The rules of the Dispute have now changed. All communication must be done through the Mediator.
    It is at this point that the business logic specific evidence-gathering can be applied, so that
    the artifical intelligence in the module can provide a second opinion to the Mediator.
    The Mediator, being a specialised and trained individual, can choose to ignore or amend the given
    advice.
  
  Background:
    Given the Dispute is fully underway
    And the Dispute is in Mediation

  Scenario: Block Agent A and B from communicating with one another
    Given we have not activated round-table communication
    Then I should not be able to communicate with the other Agent

  Scenario: Mediator requires further information
    Given a Dispute type was selected at the beginning of the Dispute # e.g. "maritime collision"
    Then the type-specific module should offer custom forms to the Agents to fill in

  # business logic specific stuff relating to Maritime Collisions etc MUST be put into a separate feature file
  # (in the plugin directory). This set of features must be as abstract and generic as possible.

  Scenario: Filling in the type-specific forms
    Given I am an Agent
    And I have filled in the forms provided by the type-specific module
    Then there should be no more forms to fill in
    And I should see that the system is awaiting a response from the other Agent

  Scenario: Filling in the type-specific forms as the second Agent
    Given I am an Agent
    And I have filled in the forms provided by the type-specific module
    And the other Agent has also filled in the forms
    Then there should be no more forms to fill in
    And I should see that the system is awaiting a response from the Mediator

  Scenario: AI logic is applied
    Given I am a Mediator
    And both Agents have completed the type-specific module forms
    Then I should see the results of the AI in the type-specific module
    #And I should be able to advise each Agent individually
    # Commented out the above line because it isn't testable. Essentially, the Mediator can send a
    # private message to either Agent, negotiating a resolution. It is up to the Agents to formally
    # send an offer through the "Propose Resolution" facility.

  Scenario: Accepting the Mediator's offer
    Given the Mediator has given me an offer
    Then I should be able to accept the offer
    And the Dispute should close successfully

  Scenario: Declining the Mediator's offer
    Given the Mediator has given me an offer
    Then I should be able to decline the offer
    And the Dispute should remain open

  Scenario: Sending an offer for round-table communication
    Given I am a Mediator
    Then I should be able to offer round-table communication
    # The Mediator should (through a dedicated facility) be able to propose round-table negotation,
    # whereby the free communication of all parties is enabled.

  Scenario: Accepting the offer for round-table communication
    Given the Mediator has suggested round-table communication
    Then I should be able to accept the offer
    And the Dispute should go into Round Table Mediation mode

  Scenario: Declining the offer for round-table communication
    Given the Mediator has suggested round-table communication
    Then I should be able to decline the offer
    And the Dispute should remain open and under Mediation
\end{lstlisting}

\section{Amended Requirements}

As the project progressed, some of the original requirements were simplified or removed altogether, and other requirements arose. The agile nature of this project meant that the features were able to evolve naturally over time.

\subsection{Account Creation}

\begin{lstlisting}
Feature: Account Creation
    I should be able to register an account of a certain type, e.g. Law Firm/Agent
    And I should be able to log into said account

  Scenario Outline: Not all fields filled in on registration
    When I fill in the details for a new Organisation account
    # (which could be Law Firm or Mediation Centre)
    And I leave the '<field_label>' field blank
    And I try to register
    Then I should see the message 'Please fill in all fields.'

    Examples:
      | field_label       |
      | Email             |
      | Password          |
      | Organisation Name |

  Scenario: Trying to register with an email address that already exists
    When I fill in the details for a new Organisation account
    And I provide an email that is already registered to the system
    And I try to register
    Then I should see the message 'An account is already registered to that email address.'

  @clear
  Scenario: Law Firm registration
    When I fill in the details for a new Organisation account
    And I try to register
    Then the account should be created
    # As discussed in the supervisor meeting, we could add an admin verification stage after account creation.
    # This could be as complicated as we want to make it, so for now, let's add a boolean in the database that
    # says if the Law Firm is verified or not. Make it verified by default, but carry out isVerified checks at login.
    # We can add the verification steps later and make Companies unverified by default.
    # At that stage, we need to add additional scenarios, e.g.
    # Given I am unverified, When I log in, Then I should have restricted access

  Scenario: Account login with valid credentials
    Given I have registered an Organisation account
    When I attempt to log in with valid credentials
    Then I should be logged into the system

  Scenario: Account login with invalid credentials
    Given I have registered an Organisation account
    When I attempt to log in with invalid credentials
    Then an authentication error should be displayed

  Scenario: Create Agent account
    Given I am logged into a Law Firm account
    Then I should be able to create an Agent account
    And I should be able to log into that account

  Scenario: Create Mediator account
    Given I am logged into a Mediation Centre account
    Then I should be able to create a Mediator account
    And I should be able to log into that account
\end{lstlisting}


\subsection{Administrator}

\begin{lstlisting}
Feature: Admin
    SmartResolution administrators can log into the instance and
    customise the instance according to the needs of the business.

  Scenario: Logging into an admin account
    Given I am logged into an Admin account
    Then I should see admin-only options on the dashboard

  Scenario: Installing a new module
    Given I am logged into an Admin account
    When I visit the SmartResolution Marketplace
    Then I should be able to install new modules

  Scenario: Activating the module
    Given I am logged into an Admin account
    And I have installed a new module
    Then I should be able to activate the module

  Scenario: Deactivating a module
    Given I am logged into an Admin account
    And I have activated a module
    Then I should be able to deactivate the module

  Scenario: Uninstalling a module
    Given I am logged into an Admin account
    And I have installed a new module
    Then I should be able to uninstall the module

  # # This could be added in a later version of SmartResolution, when the theme-customisation feature is created.
  # Scenario: Customising the SmartResolution instance
  #   Given I am logged into an Admin account
  #   Then I should be able to change the instance logo
\end{lstlisting}

\subsection{Dispute Creation}

\begin{lstlisting}
Feature: Dispute creation
    Given I am logged into an authorised account
    Then I should be able to create a Dispute

  @clear
  Scenario: Creating a Dispute
    Given I am logged into a Law Firm account
    And I have created at least one Agent account
    Then I should be able to create a new Dispute

  Scenario: Attempting to create a Dispute with no Agent
    Given I am logged into a Law Firm account
    And I have created NO Agent accounts
    Then I should see the message 'You must create an Agent account before you can create a Dispute!'

  Scenario: Trying to view a dispute I shouldn't have access to
    Given I am logged into a Mediation Centre account
    When I try to view a Dispute I've not been allocated to yet
    Then I should see the message 'You do not have permission to view this Dispute!'

  Scenario: Trying to view a dispute that does not exist
    Given I am logged into a Law Firm account
    When I try to view a Dispute that does not exist
    Then I should see the message 'The dispute you are trying to view does not exist.'

  @clear
  Scenario: Initiating a Dispute against a Law Firm
    Given I have submitted a Dispute
    Then I should be able to initiate it against another Law Firm
    And I shouldn't be able to reinitiate it against a different Law Firm

  @clear
  Scenario: A Dispute is opened against my Law Firm
    Given a Dispute has been initiated against my Law Firm
    And I have created an Agent
    Then I should be able to allocate the Agent to the Dispute

  Scenario: Editing a summary
    Given I have submitted a Dispute
    Then I should be able to edit the summary

  Scenario: Editing a dispute type
    Given I have submitted a Dispute
    Then I should be able to edit the dispute type
\end{lstlisting}

\subsection{Dispute}

\begin{lstlisting}
Feature: Dispute
    The Dispute is underway, both Agents are free to communicate with one another,
    propose offers, attach evidence, etc.

  Background:
    Given I am logged into an Agent account
    And the Dispute is fully underway
    And the Dispute is not in Mediation

  @clear
  Scenario: Sending a message via an active dispute
    Then I should be able to send a message via the Dispute

  Scenario: Sending a message via an active dispute, as an unauthorised person
    Given I am logged into an Agent account that is not associated with the Dispute
    Then I should NOT be able to send a message via the Dispute

  @clear
  Scenario: Uploading evidence to a Dispute
    Then I should be able to upload evidence to the Dispute

  Scenario Outline: Attempting to view pages before you are allowed
    Given the Dispute is NOT fully underway
    When I attempt to view the '<page>' page
    Then I should see the message '<expected_message>'

  Examples:
    | page      | expected_message                              |
    | evidence  | You are not allowed to view these documents.  |
    | mediation | You do not have permission to view this page. |

  Scenario: Start the Mediation process
    Then I should be able to start the Mediation process

  @clear
  Scenario: Resolve a Dispute
    Then I should be able to mark the Dispute as resolved
    And the Dispute should close successfully

  @clear
  Scenario: Take the Dispute to Court
    Then I should be able to take the Dispute to court
    And the Dispute should close unsuccessfully

\end{lstlisting}

\subsection{Lifespan Negotiation}

\begin{lstlisting}
Feature: Negotiating a Dispute lifespan
    When a Dispute is opened and each Law Firm has allocated an Agent
    The Agents need to negotiate a Dispute lifespan
    i.e. the maximum length of time the Dispute can continue without resolution
    before being automatically taken to Court.

  @clear
  Scenario: Creating a Dispute lifespan offer
    Given both Agents have submitted the Dispute
    Then I should be able to make a lifespan offer

  @clear
  Scenario: Accepting a Dispute lifespan offer
    Given the other Agent has sent me a Dispute lifespan offer
    Then I should be able to Accept the offer
    And I should see the message 'Dispute starts in'

  @clear
  Scenario: Create a counter Dispute lifespan offer
    Given the other Agent has sent me a Dispute lifespan offer
    Then I should be able to Decline the offer
    And I should be able to make a lifespan offer

  @clear
  Scenario: Making a lifespan offer mid-Dispute
    Given I am logged into an Agent account
    And the Dispute is fully underway
    Then I should be able to make a lifespan offer
    And the Dispute should continue normally despite the renegotiation offer

  @clear
  Scenario: Renegotiating the Dispute lifespan mid-Dispute
    Given I am logged into an Agent account
    And the Dispute is fully underway
    When I make a new lifespan offer
    And the other Agent accepts the offer
    Then the new lifespan should take immediate effect
    And I should see the message 'The dispute is currently on hold until the new lifespan comes into effect.'
\end{lstlisting}

\subsection{Putting a Dispute into Mediation}

\begin{lstlisting}
Feature: Mediation
    At any point in a confirmed Dispute
    Either Agent can propose Mediation
    Whereby a Mediator is introduced to help to resolve the Dispute
    If complications arise during the Mediation Creation process - e.g. a
    list of Mediators being provided to the Agents but are not suitable,
    then either Agent can restart the Mediation process.

  Background:
    Given I am logged into an Agent account
    And the Dispute is fully underway

  @clear
  Scenario: Proposing Mediation
    Then I should be able to propose Mediation
    And my choice of Mediation Centre should be presented to the other Agent

  @clear
  Scenario: Declining a Mediation Centre proposal
    Given the other Agent has proposed a Mediation Centre
    Then I should be able to decline the proposal
    And I should see the message 'Make Mediation Proposal'

  @clear
  Scenario: Accepting a Mediation Centre proposal
    Given the other Agent has proposed a Mediation Centre
    Then I should be able to accept the proposal
    And I should see the message 'You are waiting for the Mediation Centre to provide a list of available Mediators. You will be notified when this happens.'

  @clear
  Scenario: Providing a list of available Mediators
    Given the Agents have agreed on a Mediation Centre
    And I am logged in as the Mediation Centre
    Then I should be able to provide a list of available Mediators

  @clear
  Scenario: Choosing a Mediator from the list
    Given the Mediation Centre we've agreed upon has provided a list of available Mediators
    And I am logged into an Agent account
    Then I should be able to propose a Mediator to the other Agent

  @clear
  Scenario: Declining a Mediator proposal
    Given the other Agent has proposed a Mediator
    Then I should be able to decline the proposal
    And I should see the message 'Propose Mediator'

  @clear
  Scenario: Accepting a Mediator proposal
    Given the other Agent has proposed a Mediator
    Then I should be able to accept the proposal
    And the Dispute should be in Mediation
\end{lstlisting}

\subsection{Mediation}

\begin{lstlisting}
Feature: Dispute (under Mediation)
    The rules of the Dispute have now changed. All communication must be done through the Mediator.
    It is at this point that the business logic specific evidence-gathering can be applied, so that
    the artifical intelligence in the module can provide a second opinion to the Mediator.
    The Mediator, being a specialised and trained individual, can choose to ignore or amend the given
    advice.

  Background:
    Given I am logged into a Mediator account
    And the Dispute is in Mediation

  @clear
  Scenario: Block Agent A and B from communicating with one another
    Given I am logged into an Agent account
    And we have not activated round-table communication
    Then I should not be able to communicate with the other Agent

  Scenario: Communication between Agent and Mediator
    Given we have not activated round-table communication
    Then I should be able to communicate with the Agents in individual threads
    And there should be no way for either Agent to see the messages of the other

  Scenario: Trying to send a message to someone not involved in the dispute
    Given I try to send a message to someone not involved in the dispute
    Then I should see the message 'The account you're trying to send a message to is not involved in this dispute!'

  Scenario: Enabling round-table communication
    Then I should be able to enable round-table communication

  @clear
  Scenario: Mediation is in round-table communication
    Given the Dispute is in round-table communication mode
    Then all parties should be able to communicate freely

  @clear
  Scenario: Disabling round-table communication mode
    Given the Dispute is in round-table communication mode
    Then I should be able to disable round-table communication
    And free communication should no longer be allowed between all parties

\end{lstlisting}

\subsection{Module}

\begin{lstlisting}
Feature: Module
    To extend SmartResolution with additional functionality,
    The software should support modules which can be plugged into the system.

    I've developed a Test Module specifically for this feature, so that I can
    write regression tests to make sure dashboard items are added, custom routing
    works, etc. This is for the purposes of regression testing only: by no means
    is a module REQUIRED to define custom routes and dashboard items, etc.

  Background:
    Given I am logged into an Agent account
    And the Dispute is fully underway
    And the Dispute type has been set to 'Test Module'

  @clear
  Scenario: Clear the database in preparation for this test suite

  Scenario: Dispute dashboard item added
    Then I should see a custom dashboard item

  Scenario: Custom dispute-level-routing
    When I click on the custom dashboard item
    Then I should see the message 'Hello world!'
    And this message should have been passed as a parameter rather than hardcoded
    # (this tests parameter passing)

  Scenario: Custom top-level-routing
    When I try to visit '/module-test'
    Then I should see the message 'This module is meant to test the SmartResolution module support. It adds nothing useful and should be removed from production.'
    And the page should have this selector: 'h1#test-module'
    # (this tests markdown rendering)

  Scenario: Custom module database interaction
    When I click on the custom dashboard item
    Then I should see the message 'Entries in database: 0'
    When I click on the 'Create Database entry' button
    Then I should see the message 'Entries in database: 1'
    When I click on the 'Create Database entry' button
    Then I should see the message 'Entries in database: 2'
\end{lstlisting}

\subsection{Notification}

\begin{lstlisting}
Feature: Notifications
    As a logged in user,
    When something interesting happens
    Then I should have some sort of notification so that I know about it
    And I should be able to mark that notification as read

  @clear
  Scenario: Being notified of a Dispute
    Given a Dispute has been assigned to me
    And I am logged into a one-dispute Agent account
    Then I should get a notification about the Dispute

  Scenario: Marking the Dispute as read
    Given I am logged into a one-dispute Agent account
    And I have a new notification
    When I click on the associated link
    Then the notification should be marked as read
    And the URL should be clean, with no notification parameters

\end{lstlisting}

\subsection{Profile}

\begin{lstlisting}
Feature: Profile
    Profiles are key to SmartResolution, allowing Agents to view the CVs of Mediators
    Which helps them to choose a Mediator when the Dispute requires Mediation.
    Similarly, Organisations can specify their description, linking to external sites etc.
    This helps Agents to choose a Mediation Centre in the first place.

  Background:
    Given I am logged into an Agent account

  Scenario: Viewing the profile of an Organisation
    Given I am looking at an Organisation's profile
    Then I should see the Organisation description

  Scenario: Viewing the profile of an Individual
    Given I am looking at an Individual's profile
    Then I should see the Individual's CV
    And I should see which Organisation the Individual works for

  Scenario: Editing an Organisation profile
    Given I am logged into an Organisation account
    Then I should be able to edit my Organisation's description

  Scenario: Editing an Individual profile
    Given I am logged into an Individual account
    Then I should be able to edit my CV

\end{lstlisting}

\subsection{Routing}

\begin{lstlisting}
Feature: Routing
    As this is a RESTful web app,
    Some users might try to access pages when they shouldn't.
    So we should ensure that the appropriate error messages and redirects are in place.

  Scenario Outline: Accessing session-only pages
    Given I am not logged into an account
    When I try to visit '<forbidden_page>'
    Then I should see the message 'You do not have permission to see this page. Please log in first.'

    Examples:
      | forbidden_page       |
      | /register/individual |

  Scenario Outline: Accessing Organisation-only pages
    Given I am logged into an Individual account
    When I try to visit '<organisation_only_page>'
    Then I should see the message 'You do not have permission to see this page. You must be logged into an Organisation account.'

    Examples:
      | organisation_only_page |
      | /register/individual   |
      | /disputes/new          |

  Scenario Outline: Accessing Individual-only pages
    Given I am logged into an Organisation account
    When I try to visit '<individual_only_page>'
    Then I should see the message 'You do not have permission to see this page. You must be logged into an'

    Examples:
      | individual_only_page   |
      | /disputes/8/chat       |
      | /disputes/8/close      |

\end{lstlisting}